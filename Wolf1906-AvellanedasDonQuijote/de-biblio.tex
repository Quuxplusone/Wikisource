\section{Litteratur.}

Ausser den in den Fussnoten angegebenen Spezialarbeiten wurden noch
benutzt:

{\emph Leopoldo Rius:} {\it\spanish Bibliografía crítica de las obras de Miguel de Cervantes.} 3~Bde.
Madrid 1895, 1899,~1905.

{\emph Cayetano Alberto de~Barrera y~Leirado,}
{\it\spanish Catálogo bibliográfico y biográfico del teatro antiguo español.}
Madrid~1860.

{\emph A.~Morel-Fatio} et {\emph L.~Rouanet,} {\it\french Le Théâtre Espagnol.}
Paris~1900. ({\it Bibl.\ de Bibliographies critiques}~VII.)

{\emph Georg Ticknor,} {\it Geschichte der schönen Litteratur in Spanien.} Deutsch mit
Zusätzen herausgegeben von {\emph Nicolaus Heinrich Julius}. Neue Ausgabe
Leipzig~1867. 2~Bde.\ u.~Supplementband, bearbeitet von {\emph Adolf Wolf}.

{\emph James Fitzmaurice-Kelly,} {\it\english A History of Spanish Literature.} London~1898.
Spanische Übersetzung von Adolfo Bonilla y~San~Martín. Madrid~1900.

{\emph Gottfried Baist,} {\it Die spanische Litteratur.} In Gröber's {\it Grundriss der
rom.\ Philol.} Bd.~II, Abt.~2. Strassburg~1897.

{\emph Rudolf Beer,} {\it Spanische Litteraturgeschichte.} Leipzig, Göschen 1903. 2~Bde.

{\emph Adolf Fr.~von~Schack,} {\it Geschichte der dramatischen Litteratur und Kunst in
Spanien.} Berlin~1845--46.

{\emph J.~L.~Klein,} {\it Geschichte des spanischen Dramas.} Leipzig~1871--75.

{\emph Adolf Schaeffer,} {\it Geschichte des spanischen Nationaldramas.} Leipzig~1890.

{\spanish {\it {\emph Biblioteca} de autores españoles desde la formación del lenguaje
hasta nuestros dias.} Madrid, Rivadeneyra.}

\originalpage{70}
{\emph Miguel de Cervantes Saavedra,} {\it\spanish El ingenioso hidalgo Don Quijote de la
Mancha} comentado por {\emph D. Diego Clemencín.} Madrid~1894.

{\emph A. Birch-Hirschfeld,} {\it Geschichte der französischen Litteratur.} Leipzig und
Wien~1900.

{\emph Junker,} {\it Grundriss der Geschichte der französischen Litteratur.} 4.~Aufl. Münster~1902.

{\emph Heinrich Körting,} {\it Geschichte des französischen Romans im 17.~Jahrhundert.}
Leipzig und Oppeln~1885.

{\emph André le Breton,} {\it Le Roman au XVII\sup{e} siècle.} Paris~1890.

\dots , {\it Le Roman au XVIII\sup{e} siècle.} Paris~1898.

{\emph Fr. Spielhagen,} {\it Beiträge zur Theorie und Technik des Romans.} Leipzig~1883.

{\emph Dunlop-Liebrecht,} {\it Geschichte der Prosadichtung.} Berlin~1851.

{\emph Gaspary,} Geschichte der italienischen Litteratur. Berlin~1885.

{\emph Fürst,} {\it Don Quixotes Spuren in der Weltlitteratur}
({\it Beil.\ z.\ Allg.\ Zeit.}\ 1898 No.~6).
