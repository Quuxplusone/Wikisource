
\centerline{Avellanedas DON QUIJOTE,}
\centerline{sein Verhältnis zu Cervantes}
\centerline{und seine Bearbeitung durch Lesage.}

\centerline{Von}
\centerline{MARTIN WOLF.}

\centerline{Erster Teil.}

\centerline{Avellaneda und Cervantes.}

Nachdem Cervantes neun Jahre lang auf den versprochenen zweiten
Teil seines {\it Don Quijote} hatte warten lassen, erschien im Jahre 1614
ein apokryphischer zweiter Teil unter dem Titel: {\it Segundo Tomo del
ingenioso Hidalgo Don Quixote de la Mancha por el Licenciado Alonso
Fernandez de Avellaneda.} Sicher hätte man diesem Buche in der
Folgezeit ebensowenig Beachtung geschenkt wie anderen minderwertigen
\originalpage{2}
Fortsetzungen von bedeutenden Werken, wenn nicht Cervantes selbst
in seinem zweiten Teil die Nachwelt darauf aufmerksam gemacht hätte.
Das Verbrechen des Plagiats ist jedoch für jene Zeit nicht so selten
und in unserem Falle nicht so gross, wie es nach dem Verhalten des
Cervantes und den Aussprüchen späterer Beurteiler Avellanedas scheinen
möchte. Wir sehen in dem falschen {\it Don Quijote} nur ein neues
Beispiel jener Fortsetzungsmanie, die in der spanischen Litteratur des
16.~und 17.~Jahrhunderts auftritt, und der wenig Werke von einiger
Bedeutung entgangen sind. Die {\it Celestina} des Fernando de Rojas
eröffnet mit etwa zwanzig Fortsetzungen und Nachahmungen den Reigen.
Ihr schliessen sich an der {\it Amadis de Gaula} und andere berühmte
Ritterromane, die {\it Diana} von Montemayor und der {\it Lazarillo de Tormes.}
Unter ganz ähnlichen Umständen wie der falsche {\it Don Quijote} ist eine
Fortsetzung von Mateo Alemans {\it Guzman de Alfarache} entstanden,
die~1602\pfn{1}{\piini} in Valencia unter dem Pseudonym
Mateo Luxan de Sayavedra
(d.~i.\ Iuan Marti, ein valencianischer Advokat) erschienen ist. Während
sich hier der Entlarvung des Pseudonyms infolge der deutlichen Hinweise
Alemans keine Hindernisse entgegensetzten, ist die Frage nach dem
Verfasser des Pseudo-Don Quijote bis zum heutigen Tage nicht in
befriedigender Weise beantwortet worden. Denn wir haben es auch hier
mit einem Decknamen zu tun. Da das Buch selbst keine Anhaltspunkte
bot, haben die Nachforschungen in der Hauptsache an die Vorrede
Avellanedas und an Cervantes' Äusserungen angeknüpft. So ist es auch
gekommen, dass man das Werk Avellanedas bisher noch nicht auf
seinen litterarischen Wert oder wenigstens auf seine litterarische Stellung
geprüft hat. Dies zu tun, habe ich in dieser Arbeit unternommen und
ich hoffe, dass eine Vergleichung des falschen {\it Don Quijote} mit seinem
Vorbilde auch fruchtbar für die Beurteilung des Originalwerkes sein
wird. Für die französische Litteratur hat Avellanedas Buch einige
Bedeutung, weil Lesage mit einer freien Bearbeitung desselben seine
Laufbahn als Romanschriftsteller eröffnete. Über diesen Erstlingsroman soll
der zweite Teil dieser Arbeit handeln.

\section{1. Aufnahme und Verbreitung.}

Avellanedas Fortsetzung des {\it Don Quijote} scheint keinen Erfolg
gehabt zu haben. Die Zeitgenossen schweigen vollkommen darüber;
selbst in dem umfangreichen Briefwechsel Lope de~Vegas\pfn{2}{\piinii} mit dem
Herzog von Sessa, in dem sonst alle litterarischen Ereignisse von einiger
Wichtigkeit ihren Widerhall fanden, wird sie nirgends erwähnt. Das
einzige Zeugnis über Avellanedas Buch aus dem 17.~Jahrhundert ist eine
kurze Notiz in Nicolas Antonio, {\it Bibliotheca Hispana Nova}
(1672--79)\pfn{3}{\piiniii},
die nicht viel mehr enthält, als was schon der Titel des Buches sagt.
Im 17.~Jahrhundert ist keine neue Auflage erschienen, so dass das Buch zur
\originalpage{3}
Zeit Lesages\pfn{1}{\piiini} schon äusserst selten war. Es ist mir daher
unbegreiflich, wie Puisbusque\pfn{2}{\piiinii}
fragen kann: {\itquoted Pourquoi la prétendue suite
d'Avellaneda, bien que remplie d'invectives contre l'auteur, a eu plus
de succès que l'ouvrage même?}

Im ganzen sind sechs {\emph Ausgaben in spanischer Sprache}
erschienen.\pfn{3}{\piiiniii}

\originalpage{4}
In das Französische ist der {\it Don Quijote} Avellanedas
zweimal\pfn{1}{\pivni}
und in das Englische drei- oder vielleicht richtiger nur zweimal
{\emph übersetzt} worden\pfn{2}{\pivnii}.

\originalpage{5}
Von der {\emph französischen Bearbeitung des Lesage} gibt
es sechs Ausgaben und ausserdem Übersetzungen ins Englische,
Holländische und Deutsche\pfn{1}{\pvni}.

Daraus geht hervor, dass Avellanedas Fortsetzung nur in der
Fassung von Lesage einige Aufnahme gefunden hat. Von den sechs
spanischen Ausgaben gehen drei aus rein wissenschaftlichem Interesse
hervor. Gleichwohl finden wir in der französischen Litteratur nirgends
Spuren von Lesages Buch.

Eine Übersetzung von Lesages {\it Don Quichotte} muss auch in Popes
Hände gelangt sein und ihm Stoff zu einer Bemerkung in seinem
{\it Essay on Critricism}\pfn{2}{\pvnii} geliefert haben. Für eine Stelle in
Sternes {\it Sentimental Journey}\pfn{3}{\pvniii},
die Becker\pfn{4}{\pvniv} anführt,
ist das Vorbild bei Lesage Cap.~10 zu suchen.

\section{2. Die Verfasserfrage.}

Die häufigen Anspielungen, die Cervantes vom 59.~Kapitel des
zweiten Teiles ab auf Avellaneda macht, und die Leidenschaftlichkeit,
mit der er auf seine Angriffe antwortet, mussten jeden Leser einer
\originalpage{6}
späteren Zeit stutzig machen. Die Zeitgenossen wussten vielleicht ganz
genau, mit wem sie es zu tun hatten. Seit es eine Cervantesbiographie
gibt, ist daher auch seinem Nachahmer besondere Aufmerksamkeit zu
teil geworden. Nachdem es einmal feststand, dass man ein Pseudonym
vor sich hatte, --- es hätte dazu nicht erst der Beweise von
Isidro Perales\pfn{1}{\pvini}
bedurft ---, trat die Frage nach dem wahren Verfasser
des unechten zweiten Teiles in den Vordergrund. Denn von ihrer
Lösung erhoffte man auch Aufschlüsse über die litterarischen Streitigkeiten
des Cervantes und die im ersten Teil enthaltenen Anspielungen. Aber
trotz der nunmehr etwa hundertundsechzig Jahre währenden Bemühungen
der Cervantisten, den Plagiator zu entlarven, ist bis jetzt noch keine
befriedigende Lösung der Autorfrage geliefert worden.

{\emph Gregorio Mayans y Siscar}\pfn{2}{\pvinii},
der erste Biograph des berühmten Spaniers, drückt sich noch sehr
zurückhaltend aus: {\itquoted Aquellas
palabras {\rm (Vorrede zum 2.~Teil)} Señor i Grande son misteriosas para
mi; i sea lo que fuere: Yo estoi persuadido á que el enemigo de
Cervantes era mui poderoso, quando un Escritor, Soldado, animoso i
diestro en el manejo de la pluma i de la espada, no se atrevió á
nombrarle. Si ya no es que fuese hombre tan vil, i despreciable, que ni
aun quiso que se supiese su nombre para que con la misma infamia
no lograse alguna fama.} Wenn Cervantes Avellaneda {\it señor autor}
nennt, so drückt das kaum Hochachtung aus, {\it Grande} nennt er ihn
dagegen nirgends; es scheint mir dies auf einem Missverständnis zu
beruhen (--- {\it la (affliccion) que debe tener este señor sin duda es grande\dots})

Hatte der {\emph Padre Murillo} in seiner {\it Geografía histórica}~(1752)
behauptet, dass Avellaneda ein Geistlicher gewesen sein müsse, geht
{\emph Vicente de los Ríos}\pfn{3}{\pviniii}
weiter und meint, er, müsse als Geistlicher
die dritten Weihen empfangen haben. Ausserdem nimmt er den aragonesischen
Ursprung nach {\it D.~Q.}\ Cap.~59 als sicher an und vermutet in
ihm einen Komödiendichter, der sich durch eine versteckte Kritik seiner
Komödien im 1.~Teil verletzt gefühlt habe. Erstere Annahme sucht
{\emph Pellicer}\pfn{4}{\pviniv}
noch durch einige Beispiele aus der Sprache des Buches
\originalpage{7}
besser zu begründen. Ferner behauptet er, dass Avellaneda ein Dominikaner
war, wegen der häufigen Erwähnung dieses Ordens in seinem
Buch und wegen der hervorragenden Rolle, die der Marienkult und die
Rosenkranzverehrung darin spielen. In zwei {\it vejámen,} die im Codex
Fernan Nuñez enthalten sind und einem {\it certámen} entstammen, das am
16.~Oktober~1614 bei Gelegenheit der Seligsprechung der Santa Teresa
de~Jesús in Zaragoza stattfand, glaubte er Anspielungen auf Avellanedas
Werk zu finden. Indessen ist das wenig wahrscheinlich. Vielmehr wird
es sich auf einen Studenten\pfn{1}{\pviini}
beziehen\pfn{2}{\pviinii}, der in der Maske Sancho
Panzas an dem {\it certámen} teilgenommen und keinen Preis erhalten hatte.
Es lautet: {\itquoted Sancho Panza, estudiante, \vbar Oficial ó pascante, \vbar
Cosa justa á su talento, \vbar Le dará el verdugo ciento\pfn{3}{\pviiniii}, \vbar
Caballero en Rocinante.} Ganz ähnlich ist der Inhalt des anderen Spottgedichtes. Die
verhältnismässig schonende Behandlung, die Cervantes seinem Gegner
angedeihen lässt, sucht Pellicer dadurch zu erklären, dass Avellaneda
sich als Dominikaner und Aragonese der Protektion des Beichtvaters
des Königs, Fray Luis de Aliaga, erfreut habe\pfn{4}{\pviiniv}.

Somit waren einige Punkte festgelegt, die zur Bestimmung des Verfassers
dienen konnten. Für die nun zu besprechenden Hypothesen, die
auf eine bestimmte Persönlichkeit abzielten, kann man unter Benutzung
des eben behandelten Materials folgende Leitsätze aufstellen:

\vskip 0.5em

\begin{hangparas}{\myitemhang}{1}
1. Avellaneda muss im 1.~Teil in irgend welcher Weise verletzt
worden sein (vgl.\ seine Vorrede).

2. Diese Beleidigung ist enthalten in irgend welcher persönlichen
Anspielung {\it (sinónimos voluntarios?)}\/, in dem {\it escrutinio}
({\it D.~Q.}\ I,~6) oder in der Kritik der Komödie ({\it D.~Q.}\ I,~48).

3. Er war Komödiendichter, mindestens aber Schriftsteller.

4. Er war ein Freund Lope de Vegas (vgl.~Vorrede Avellanedas).

\originalpage{8} 5. Er muss, wenn er nicht Mönch war, wenigstens
Theologie studiert haben (cf.~Pellicer, a.~a.~O.\ S.~159~ff.).

6. Er war Aragonier ({\it D.~Q.}\ II,~59).

7. Er kannte Zaragoza und Alcalá (Av. Cap.~22,~23 u.~26).
\end{hangparas}

\vskip 0.5em

Seit {\emph Cean Bermudez} als erster im Jahre 1808 eine bestimmte
Persönlichkeit als den Verfasser des {\it Don Quijote} bezeichnet hat, nämlich
den Pater Blanco Paz aus Estremadura, sind eine ganze Menge neuer
Hypothesen zum Vorschein gekommen. Da diese zum Teil von selbst
fallen (Luis de Granada gest.~1595, Pedro Liñán gest.~1609, Gabriel
Tellez, der bis~1613 überhaupt nicht als Schriftsteller hervorgetreten ist),
zum Teil auch von anderer Seite widerlegt worden sind (Alarcon, Argensola,
Andrés Perez etc.) können wir uns hier auf solche beschränken
die entweder durch ihren jüngeren Ursprung oder durch die mehrfache
Unterstützung, die sie gefunden haben, eine besondere Besprechung
heischen.

Am erfolgreichsten war die viel umstrittene {\emph Aliagatheorie}, die
1846 von {\emph Adolfo de Castro}\pfn{1}{\pviiini}
aufgestellt wurde und trotz des~1872
von Tubino\pfn{2}{\pviiinii} gelieferten Gegenbeweises
in neuerer Zeit so an Bedeutung
gewonnen hat, dass Don José Maria Asensio\pfn{3}{\pviiiniii}~1901 sie mit neuen
Beweisen zu stützen suchte. Bisweilen wird daher die Autorschaft Aliagas
als vollkommen sicher hingestellt\pfn{4}{\pviiiniv}. Als ihre Anhänger haben sich auch
La Barrera\pfn{5}{\pviiinv}
und Don Cayetano Rosell\pfn{6}{\pviiinvi} bekannt. Die Veranlassung
zu der Hypothese Adolfo de Castro gab folgendes Gedicht des Grafen
Villamediana\pfn{7}{\pviiinvii}:

\vskip 0.5em

{\obeylines\parindent=2in%
\glqq Sancho Panza, el confesor
Del ya difunto monarca,
Que de la vena del arca
Fué de Osuna sangrador,
El cuchillo de dolor
Lleva á Huete atravesado
\originalpage{9} Y en tan miserable estado,
Que será, segun he oido,
De inquisidor inquirido,
De confesor confesado.\grqq
}

\vskip 0.5em

Dieses Gedicht, das zu den zahlreichen Spottversen gehört, die den
Sturz Aliagas\pfn{1}{\pixni}
im Jahre~1621 begleiteten, bezieht sich auf einen Prozess
des Herzogs von Osuna, bei dem Aliaga diesem Dienste geleistet und
sich dafür hatte bezahlen lassen\pfn{2}{\pixnii}. Die Vertreter der Aliagatheorie
behaupten, dass Aliaga den Spottnamen Sancho Panza und zwar von
Jugend geführt habe, und sehen auch in den bereits zitierten Cuadrillos
aus dem Certamen von Zaragoza Anspielungen auf ihn. Cervantes hatte
ihn dadurch beleidigt, dass er den Spottnamen des Paters seinem Schildknappen
beigelegt habe. Mit mehr Recht wird man wohl annehmen
dürfen, dass Villamediana dem Beichtvater den Namen Sancho Panzas
gegeben hat, indem er in ironischer Weise auf des Knappen uneigennützige
Verwaltung der Insel Barataria Bezug nimmt. Ob Cervantes
Aliaga, der bis zum Jahre~1602 sein Kloster in Zaragoza nicht verlassen
hat, zur Abfassungszeit des ersten Teiles gekannt hat, lässt sich nicht
feststellen. Allerdings hat er sich~1595 an einem poetischen
Wettbewerb\pfn{3}{\pixniii}
in Zaragoza beteiligt. Man weiss aber nicht, ob er auch persönlich dort
gewesen ist.

Um Aliaga als Schriftsteller ausgeben zu können, schrieb man ihm
ein pseudonymes Pamphlet gegen Quevedo zu: die {\it l'enganza de la lengua
española contra el autor del Cuento de Cuentos, por Don Juan Alonso
Laureles, caballero de habito y peon de costumbre, aragones liso y
castellano revuelto.} Huesca~1629\pfn{4}{\pixniv}.

Auf die Frage, ob überhaupt stilistische Ähnlichkeiten zwischen
dieser Schrift und dem {\it Don Quijote} von Avellaneda bestehen, werde
ich später zurückkommen. Die Verfasserschaft Aliagas ist schon deshalb
unmöglich, weil diese Streitschrift frühestens~1628 verfasst ist,
der Exbeichtvater aber bereits~1626 gestorben ist. Diese Datierung verlangen
\originalpage{10}
die Anspielungen auf Quevedos {\it Sueños}\pfn{1}{\pxni}~(1627) und auf seinen
{\it Discurso de todos los Diablos}~(1628)\pfn{2}{\pxnii}.
La Barrera glaubt in einer Episode des
{\it D.~Q.}\ (II,~61) eine Anspielung auf Aliaga (Stachelginster \longeq aliaga) zu
sehen. Endlich hat Asensio aus den Anfangsworten des Avellaneda'schen
Werkes ein Anagramm herausgelesen: {\itquoted El sabio A~l~i~solan, historiador
no menos moderno que verdadero, dice que siendo expelido los moros
a~g~a~renos \dots} Schliesslich hat ein Anagramm den Zweck, dass es dem
Leser sofort in die Augen fällt. Das tut dieses aber nicht. Was endlich
die Verfasserschaft Aliagas vollkommen unmöglich erscheinen lässt, ist
seine Stellung. In den Jahren 1609--1613, wo ein grosser Teil der
Regierungsgeschäfte auf seinen Schultern ruhte, wird er kaum Zeit gehabt
haben, einen Roman zu schreiben. Ihm standen ja auch andere Mittel
zu Gebote, wenn er sich für irgend etwas an Cervantes hätte rächen
wollen. Ich glaube, dass sich aus dieser vielleicht etwas flüchtigen Prüfung
der Hypothese Adolfo de~Castros ergeben hat, dass der Beichtvater Luis
Aliaga nicht mit dem Verfasser des falschen {\it Don Quijote} identisch
sein kann.

Gewisse Ähnlichkeit mit dieser Hypothese in der Verwendung des
Beweismaterials hat eine andere, die von {\emph Marcelino Menendez y
Pelayo}\pfn{3}{\pxniii} aufgestellt wurde.
Unter den im Zaragozaer Certámen genannten Dichtern wählte er sich einen aus:
{\emph Alonso Lamberto}, der
sich schon deshalb für seine Zwecke eignete, weil man nicht das Geringste
von ihm weiss. Aus den Buchstaben der ersten vier Wörter des Buches
setzt er sich den Namen zusammen und erklärt auch Solisdán ({\it De Solisdán
á Don Quixote}, Soneto) als eine Umstellung von Don Alonso. Groussac\pfn{4}{\pxniv}
wird wohl das Richtigere getroffen haben, wenn er in Solisdán ein
Anagramm von Lasindo\pfn{5}{\pxnv}
sieht. Ein näheres Verhältnis Lambertos zu Lope
de~Vega ist ebenso wenig erwiesen, wie seine Verwandtschaft mit Don
Martin Lamberto Iñigues, die übrigens nicht genügen würde, um ihn als
berufsmässigen Schriftsteller zu kennzeichnen.

\originalpage{11}
Menendez y Pelayo hält es nicht für unmöglich, dass auch die
bereits erwähnte {\it Venganza de la lengua española} von Alonso Lamberto
stammt. Da müsste man doch erst einmal feststellen, ob wirklich
Stilähnlichkeiten\pfn{1}{\pxini}
zwischen Laureles und Avellaneda bestehen. Die {\it Venganza}
ist ein Erzeugnis der aus streng religiösen Kreisen hervorgehenden
Opposition gegen Quevedo, die noch mehr Streitschriften\pfn{2}{\pxinii}
gegen ihn
geliefert hat. Die Polemik gegen die sprachlichen Bestrebungen dieses
Satirikers im {\it Cuento de Cuentos}~(1626), der den Kampf gegen die
sprichwörtliche Ausdrucksweise im Spanischen wiedereröffnet, den der Verfasser
im Jahre~1600 mit einer {\it Pregmática}\pfn{3}{\pxiniii}
begonnen und in der {\it Visita de los
chistes} fortgeführt hatte, dienen dem Pamphletisten nur als Vorwand für
andere Anklagen. In philologischer Hinsicht dürfen wir ihm allerdings
Recht geben. Doch, er hält sich nicht lange mit diesen Fragen auf, bald
geht er, seiner eigentlichen Absicht entsprechend, zum Kernpunkt seiner
Gegnerschaft über, indem er darauf aufmerksam macht, welche Gefahr
in moralischer\pfn{4}{\pxiniv}
und religiöser Beziehung die Werke Quevedos durch
die Verspottung des geistlichen Standes\pfn{5}{\pxinv}
in sich bergen. Die ganze
schlau berechnete Art seiner Polemik, bei der er immer wieder darauf
zurückkommt, dass er dem Autor im Grunde wohl wolle und ihm anrate,
etwas vorsichtiger zu sein und Lope de~Vega nachzuahmen, ist grundverschieden
von dem unvorsichtigen, täppischen Zuschlagen Avellanedas.
Die angebliche Ähnlichkeit in dem Stile der beiden Schriftsteller habe
ich nicht finden können, zumal da nicht gesagt wird, worin sie eigentlich
\originalpage{12}
bestehen soll. Vielmehr hat die {\it Venganza} eine Stileigenheit, die ich an
Avellaneda nicht bemerkt habe, ich meine das Spielen mit gleichlautenden
Worten verschiedener Bedeutung und verschiedenen Worten gleichen
Stammes~z.~B.: {\itquoted me duele su tentada flaqueza, desatentada lengua, y
papeles hechos á tiento de pintor} etc.~und {\itquoted mire que es religioso, y debe
ser sacrolego, pero no sacrilego}. Ich kann daher nicht an die Identität
des Verfassers der {\it Venganza} mit dem Verfasser des falschen {\it Don Quijote}
glauben.

Somit müssen wir auch die Hypothese Menendez y~Pelayos, auf
die er übrigens selbst nicht viel Gewicht legt, aufgeben.

Einen Fortschritt in der Methode bedeutet die von Groussac\pfn{1}{\pxiini}
gegebene Lösung der Verfasserfrage. Er stützt sich auf Beobachtungen
sprachlicher Natur, die ihm dazu dienen sollen, nachzuweisen, dass
Avellaneda und {\emph Juan Martí,} der Verfasser des falschen {\it Guzman}, ein und
dieselbe Person sind. Die bereits von Pellicer angeführten Aragonismen
und einige neue Beispiele gibt er als Valencianismen aus und sucht
dafür Entsprechungen bei Juan Martí. Pellicer\pfn{2}{\pxiinii} bezeichnet folgende
Ausdrücke als aragonesisch: {\itquoted Califica {\rm (Cervantes)} el lenguaje
{\rm (de Avellaneda)} de aragonés, porque tal vez escribía sin articulos, y pudiera
haber alegado otras pruebas no menos convincentes que copiosas como
son: 1. en salir de la carcel, por en saliendo ó habiendo salido; 2. á la
que volvió la cabeza, por habiendo vuelto la cabeza; 3. escupe y le
pegaré, por le castigaré; 4. hincar carteles, por fixar ó pegar; 5. poner
la escudilla en las brasas, por poner la taza sobre las asquas; 6. el señal,
por la señal; 7. menudo, por mondongo; 8. mala gana, por congoja,
desmayo ó vaguido, y 9.~aquel tratarse las personas de impersonál como
\glq{}mire, oyga, perdone\grq.}

Das von Cervantes angeführte Kriterium für den aragonesischen
Ursprung kann nicht allzu schwer ins Gewicht fallen, wenn man die
Fälle betrachtet, in denen Avellaneda den Artikel auslässt: {\itquoted ello es verdad
que no todas (las) veces nos salían las aventuras como nosotros
querían} \dots und {\itquoted con esto hacia toda (la) resistencia que podía para
soltarse.} Im Katalanischen\pfn{3}{\pxiiniii} ebenso wie im älteren
Spanisch kommt todo ohne Artikel vor: {\itquoted en todas guisas} {\it (Amadis)},
{\itquoted porque en invierno
\originalpage{13}
no es menester fresco, y en verano no lo hay todas veces} (P.~de~Mejía,
{\it Coloquio del porfiado\/}), {\itquoted no podian ejecutar las temas de sus locuras
todas veces} (Quevedo, {\it Casa de locos de amor\/}). Eher als durch diesen
Gebrauch wird Cervantes durch den Druckort Tarragona auf die Vermutung
gekommen sein, dass der Verfasser Aragonese war.

Von den sonst bei Pellicer angeführten Beispielen sind {\it escudilla,
menudo, pegar, brasas} im Wörterbuch der Akademie als castilianisch
verzeichnet. Die unpersönliche Anrede gehört nach Borao\pfn{1}{\pxiiini}
der Mönchssprache an. Es bleibt also noch zu behandeln:

1. {\it en} mit Inf.\ für {\it al} mit Inf. Ausser dem Beispiel Pellicers noch:
{\itquoted\spanish y la primera cosa que hizo en despertar fué preguntar á Sancho por
la reina Cenobia}~(86\sup{b}). Sonst finden wir stets {\it al} (5\sup{a}, 11\sup{b}, 22\sup{b},
27\sup{b}, 82\sup{b} etc.). In dem falschen {\it Guzman} bemerken wir dasselbe
Schwanken zwischen {\it en} und {\it al}. Morel-Fatio meint, dass die zwei Fälle
{\it en} gegenüber den vielen Fällen {\it al} bei Avellaneda auf Rechnung des
aragonesischen Druckers gesetzt werden könnten.

2. {\it á la que \dots} ist allerdings eine Besonderheit, die ich sonst bei
keinem Schriftsteller jener Zeit gefunden habe. Es steht elliptisch für
{\it á la ora que.} Ob im Katalanischen etwas Ähnliches vorkommt, ist mir
nicht bekannt. Das von Groussac angeführte Beispiel aus Juan Martí
(364): {\itquoted\spanish á cuatro que le refieren} etc. gehört nicht hierher.

3. {\it señal} als m.\ findet sich einmal bei Avellaneda und würde ein
Katalanismus sein ({\it lo senyal\/})\pfn{2}{\pxiiinii}. Sonst sagt er:
{\itquoted\spanish la señal acostumbrada}
(60\sup{a}), {\itquoted\spanish me hize esta señal en el rostro} (69\sup{b}),
{\itquoted con una señal de una
espada de fuego \dots otra señal parda de color de acero} (26\sup{b}) etc.
Nichts bei Juan Martí.

4. {\it mala gana} \longeq Unwohlsein ist ein Katalanismus: einmal bei
Avellaneda (97\sup{a}) und zweimal bei Juan Martí (373\sup{b} und 376\sup{b}).

Das Verzeichnis Pellicers hat Groussac noch um einiges bereichert:
{\spanish{\it zorriar}, {\it buen recado} (bastante), {\it repostona}, {\it otorgar}
(confesar), {\it aun} (así),
{\it aunque} (puesto que), {\it hendo} (haciendo), {\it repapo}, {\it pedir de}
(preguntar por), {\it partera} (parida).}

1. {\it zorriar} \longeq zurriar (der Wechsel von {\it o} und {\it u} ist nicht wesentlich)
ist castilianisch und findet sich auch bei Quevedo. Ich habe es in
keinem katalanischen Wörterbuch gefunden. Nicht bei Juan Martí.

\originalpage{14}
2. {\it repostón, --- a} (8\sup{a}, 65\sup{a}, 83\sup{b} und 110\sup{b}) Borao:
{\itquoted Hemos oido muchas veces esa palabra, usada hoy sin distincia de clases.}
Ich habe es aber in keinem katalanischen Wörterbuch gefunden.

Die Bildung ergibt sich leicht aus {\it repuesto} + Suffix {\it -ón.}

2. {\it á buen recado} hat nicht den Sinn von {\it bastante\/}: {\it ¡Buen recado
se tiene!} (111\sup{a}) heisst: Der ist unter sicherm Verschluss. Übrigens
ist es gut kastilianisch.

3. {\it otorgor, aun} und {\it aunque} in den von Groussac angegebenen
Bedeutungen sind nicht auffindbar.

4. {\it Hendo} und {\it her\/}: in der familiären Sprache von ganz Spanien,
werden auch von Cervantes gebraucht.

5. {\it repapo\/}: {\itquoted Sancho durmió aquella noche muy de repapo} (15\sup{a}).
Borao citiert es, kennt aber keine sonstigen Belege. Wenn wir es nicht
als eine Neubildung zu {\it papo} (essen \longeq {\it henchir lo papo,} kat.~{\it omplir lo
pap,} oder {\it de referto papo\/}) ansehen, könnte man es zu kat.~{\it repaparse \longeq
repantigarse} stellen. Dann würde es allerdings ein Katalanismus sein.

6. {\it pedir de\/}: findet sich nicht bei Avellaneda, dafür aber oft
{\it preguntar por} (13\sup{b}, 25\sup{b}, 41\sup{b}, 43\sup{b}, 46\sup{a} etc.).
Wenn es vorkäme, wäre es auch kein Katalanismus\pfn{1}{\pxivni}.

7. {\it partera} für {\it partida\/}: Katalanismus. {\itquoted Se diu de la femella que
fa poch temps ha parit.} (Labernia y~Estellez.)

8. {\it henchir \longeq llenar} ist kastilianisch (Covarrubias).

9. {\it buena voya\/}: italienisches Lehnwort ({\it buona voglia} --- der
Freiwillige auf der Galeere).

So bleiben als katalanisch {\it partera, mala gana} und vielleicht {\it repapo.}
Trotzdem ist nicht ausgeschlossen, dass Avellaneda ein Aragonese war.
Mir kam es nur auf den Beweis an, dass die genannten Besonderheiten
zu einem Identitätsnachweis unbrauchbar sind.

Dagegen hat Avellaneda eine Vorliebe für gewisse Wendungen,
die wir wieder nicht bei Juan Martí finden z.~B.\ {\it ello es verdad, no
entender la música, hacer pelillos á la mar} etc. Besonders charakteristisch
für ihn ist, wie schon Morel~Fatio bemerkt, der ausgiebige
Gebrauch, den er von der Praeposition {\it tras} macht. Das Verhältnis der
Anwendung dieser Praeposition bei beiden Schriftstellern stellt sich,
wie folgt (zu beachten ist, dass der falsche {\it Guzman} um ein Drittel
kürzer ist als der {\it Don Quijote\/}):
\originalpage{15}

\halign{# & #\hfil & \hfil # & \hfil # & # \cr
       &                  & Avellaneda & Juan Martí \cr
Praep. & tras             & 54 &  19 & Fälle \cr
''     & tras de          &  2 &   4 & '' \cr
       & tras esto        & 42 &   5 & '' \cr
       & tras lo cual     & 29 & --- & '' \cr
       & tras que         & 13 & --- & '' \cr
       & tras c.~Inf.\    &  7 & --- & '' \cr
       & tras de c.~Inf.\ &  1 & --- & '' \cr
}

Andrerseits hat Juan Martí eine besondere Vorliebe für die Formel
{\it aunque --- pero} (resp.~{\it empero}), die er 24~mal anwendet gegen keinmal
bei Avellaneda, z.~B.\ {\itquoted\spanish Y aunque yo tampoco miraba por el mio {\rm (provecho),}
pero tenia hecha costumbre de casa de monseñor.}

Avellaneda bevorzugt ferner Satzanschlüsse durch ein Demonstrativum:
neben {\it tras esto} auch {\it con esto, en esto, oyendo esto, con esta
quimera} etc. Eine stilistische Verwandtschaft zwischen beiden
Schriftstellern ist demnach ausgeschlossen.

Aber, wie Groussac bemerkt, dürfen wir nicht allzu viel Vertrauen
in eine sprachliche Vergleichung setzen, da sich der Stil eines Menschen
in der Zeit von zehn Jahren bedeutend ändern kann. Er sucht daher
die Reflexe derselben Persönlichkeit in beiden Werken nachzuweisen.
Gewisse Zitate scheinen ihm auf der Grundlage einer gleichen Bildung
zu beruhen. Mir macht es aber den Eindruck, als ob Juan Martí in
seinen überreichlichen Zitaten mehr Gelehrsamkeit zeigt als Avellaneda
mit seinen paar Fetzchen Latein, die nicht einmal überall richtig sind
({\itquoted\latin parcere prostratis docuit nobis --- {\rm statt} nos --- ira leonis};
der Ausspruch: {\itquoted\latin est deus in nobis}\pfn{1}{\pxvni}
ist nicht von Horaz sondern von Ovid).
Die zweimalige Nennung der Cenobia im {\it Guzman} soll schon auf die
Cenobia des {\it Don Quijote} hindeuten, eine einmalige Erwähnung Lope
de Vegas auf die Freundschaft Avellanedas mit ihm, ebenso die
ausführlichen Betrachtungen über Gefängnis und Theater auf die spärlichen
Urteile über die gleichen Gegenstände bei Avellaneda. Die Kenntnis
des weltlichen und kanonischen Rechtes, die Avellaneda wie Juan Martí
besitzen soll, beschränkt sich auf eine einmalige Nennung der {\it bula de
composicion.} Den falschen {\it Guzman} beherrscht ein ganz anderer Geist
als den Pseudo-{\it Don Quijote.} Nichts liegt Avellaneda ferner als jene
Lust am Philosophieren und Moralisieren, die Juan Martí oft zu
\originalpage{16} seitenlangen
Betrachtungen über alle möglichen Dinge veranlasst. Avellaneda
ist knapper in der Darstellung und familiärer im Ausdruck. So wird
man bei beiden mehr Verschiedenheiten als Berührungspunkte finden.
Wir müssen also auch über die Hypothese Groussacs den Stab brechen
und kommen zu dem letzten zu besprechenden Lösungsversuch der Verfasserfrage.

Avellaneda sagt in der Vorrede, dass er auch für die Beleidigungen,
die von Cervantes Lope de Vega zugefügt wären, an ihm Rache nehmen
wolle. Sollte sich nicht etwa {\emph Lope de Vega} selbst hinter dem
Pseudonym verbergen? Bei der Rivalität zwischen beiden Dichtern, die jeden
mit dem andern auf dessen Gebiet wetteifern liess, Lope de Vega in
der Prosa, Cervantes im Lustspiel, lag der Gedanke daran sehr nahe.
Aber es spricht vieles von vornherein gegen diese Hypothese, die von
{\emph Mainez}\pfn{1}{\pxvini} aufgestellt wurde. Erstens muss sich das Verhältnis zwischen
beiden Dichtern, das beim Erscheinen des ersten Teiles des {\it Don Quijote}
äusserst gespannt war, mit der Zeit etwas gebessert haben. Denn im
Jahre~1612 finden wir beide in der Academia Selvaje. Lope de Vega
erzählt selbst in einem Brief an den Herzog von Sessa, dass er sich bei
einer Akademiesitzung von Cervantes eine Brille zum Lesen eines Gedichtes
geborgt habe, deren {\quoted Gläser ausgesehen hätten wie ein paar
schlechtgemachte Spiegeleier}\pfn{2}{\pxvinii}. Auch in ihren späteren Werken sprechen
beide nur im Tone der grössten Hochachtung von einander. Zweitens sind
auch nicht die geringsten Ähnlichkeiten in dem Stile Lopes und Avellanedas
vorhanden, obgleich Mainez das Gegenteil behauptet\pfn{3}{\pxviniii}. Ich traue
mir nicht zu, über diesen Punkt treffender zu urteilen, als dies Menendez
y~Pelayo getan hat: {\quoted Dass Lope der Verfasser des {\it Don Quijote} von
Avellaneda ist, ist höchst unwahrscheinlich. Der so charakteristische Stil
dieses Romans hat nichts mit der Manier zu tun, die Lope als Prosaist
hatte. Er ähnelt weder der poetischen und latinisierten Prosa der {\it Arcadia}
oder des {\it\spanish Peregrino en su patria,} noch der ungezwungenen und eleganten
historischen Prosa des {\it\spanish Triunfo de la fé en los reinos del Japón,} noch
der angenehmen, natürlichen, ausdrucksvollen und anmutigen Diktion
vieler Scenen der {\it Dorotea,} die bisweilen mit der {\it Celestina} zu wetteifern
\originalpage{17}
sucht, schliesslich auch nicht dem possenreisserischen Witz der
Privatbriefe, die, wenn sie auch den Verfasser wenig ehren, von grossem Wert
für die Beurteilung seines Geistes und Humors sind. Aber auch in dieser
privaten Korrespondenz, wo der grosse Dichter oft ohne Anmut alle
Schranken durchbricht, gibt es nichts, was der unanständigen Plumpheit
Avellanedas gliche. Wenn er für das Publikum schreibt, verfährt er,
sobald er Bilder von schlechten Sitten zeichnet, die in seinem ungeheuren
Theater nicht fehlen durften, wenn es wirklich ein vollkommenes Abbild
der menschlichen Komödie sein sollte, mit einer gewissen Sparsamkeit
und einem guten Geschmack, den Avellaneda nie besass. So in der
{\it Dorotea} selbst, in dem {\it Ansuelo de~Feniza,} dem {\it Rufián Castrucho,}
dem {\it Arsenal de Sevilla.} Niemals, auch nicht in der grössten Ungebundenheit,
lässt sich die edle Muse Lopes und Tirsos mit dem brutalen Realismus
Avellanedas vergleichen, der diesem allein unter allen Schriftstellern
jenes Jahrhunderts eigen ist.}

Hiermit will ich meine Erörterungen über die Verfasserfrage schliessen,
ohne dass ich mit gutem Gewissen einer der bis jetzt beigebrachten
Lösungen hätte beistimmen können. Ich bin auch nicht im stande, das
Rätsel auf eine neue Art zu lösen, da die Namen der verfügbaren
Romanschriftsteller jener Zeit bereits erschöpft sind. Wenn nicht noch neues
Beweismaterial hinzukommt, wird es kaum gelingen, eine befriedigende
Antwort auf die Frage nach dem Verfasser zu geben. Trotzdem glaube
ich nicht, dass schon das letzte Wort über Avellaneda gesprochen ist.
Denn, so oft man von dem unsterblichen Werk des Cervantes reden
wird, wird man auch mit Verachtung seinen kühnen Nachahmer nennen.
Er wird seine Ahasverrolle in der Litteraturgeschichte weiterspielen und
vielleicht noch einigemale unter neuen Namen auftauchen.

Wenn wir auch aus der Betrachtung der Verfasserfrage keinen
positiven Gewinn gezogen haben, so hat sie uns doch wenigstens einen
Vorteil gebracht. Sie hat uns darauf hingeleitet, uns ein Urteil über
den Stil und die Person des Verfassers zu bilden. Das wird uns von
besonderem Nutzen sein, wenn wir jetzt zur Prüfung des Werkes selbst
schreiten.

\section{3. Avellanedas Absichten und der Einfluss seines
Werkes auf die Vollendung des zweiten Teiles
von Cervantes' {\it Don Quijote.}}

Als Cervantes in der Vorrede zu den {\it\spanish Novelas exemplares} das
nahe bevorstehende Erscheinen des zweiten Teiles seines {\it Don Quijote}
\originalpage{18}
in Aussicht stellte, hatte sein Nachahmer, der offenbar nichts davon
wusste, dass Cervantes ebenfalls einen zweiten Teil schrieb, seine Fortsetzung
ziemlich vollendet. Es ist daher begreiflich, dass ihn die Ankündigung
des Cervantes in Aufregung versetzte. Denn der pekuniäre
Gewinn, um den es ihm zunächst ohne Zweifel zu tun war, wurde
dadurch in Frage gestellt. Da verfiel er auf das Mittel, seinem Buch
durch einen Prolog voll von dunklen Anspielungen und frechen Angriffen
auf Cervantes Zugkraft zu verschaffen. Dass es nicht in der
ursprünglichen Absicht Avellanedas lag, Cervantes in seinem Buch persönlich
zu verletzen, scheint mir daraus hervorzugehen, dass in dem
Werke selbst persönliche Anspielungen\pfn{1}{\pxviiini} jeder Art fehlen.

Cervantes wurde durch die in Avellanedas Vorrede enthaltenen
Beleidigungen schwer getroffen und antwortete mit mehr Leidenschaftlichkeit
darauf, als er nötig gehabt hätte. Am meisten fühlte er sich
dadurch verletzt, dass Avellaneda über sein Alter und seine Einarmigkeit
spottet. Denn den Arm hatte er in ehrenvoller Schlacht verloren, die
Erfahrenheit seines Alters konnte seinen Büchern nur von Nutzen sein.
Cervantes war gerade dabei, das 59.~Kapitel seines zweiten Teiles zu
schreiben, als ihm der apokryphische {\it Don Quijote} bekannt wurde.
Von diesem Kapitel an kritisiert er ihn fortwährend. Allerdings hat er
hierbei nicht die schwächsten Punkte des Buches getroffen.

Wenn Avellaneda Sanchos Frau Mari-Gutierrez nennt, so ist
Cervantes selbst daran schuld. Denn er wechselt fortwährend mit dem
Namen: bald nennt er sie Mari-Gutierrez, bald Juana Gutierrez und
erst im zweiten Teil Teresa Panza. Ebenso ist er mitschuldig, wenn
Avellaneda Sancho als gefrässig\pfn{2}{\pxviiinii} darstellt. Damit hat dieser nur eine
Eigenschaft Sanchos aufgenommen, die im ersten Teil vorübergehend
erwähnt wird. Übrigens beweist Sancho bei der Hochzeit Camachos (II,~21)
eine Esslust, die dem Sancho Avellanedas alle Ehre machen würde. Vollkommen
Recht hat aber Cervantes, wenn er dem Buch Obscönität vorwirft.

Der Pseudo-{\it Don Quijote} ist nicht ohne Einfluss auf die Schlusskapitel
des echten {\it Don Quijote} gewesen. Da Avellaneda seinen Helden
gemäss einer Andeutung, die Cervantes im letzten Kapitel des ersten
\originalpage{19}
Teiles macht, nach Zaragoza führt, musste der Dichter von seinem
ursprünglichen Plane abweichen und den Ritter nach Barcelona ziehen lassen.

Die nun folgenden Kapitel stehen nicht auf der Höhe des Übrigen.
Man sieht ihnen den Ärger und die Unlust an, die Cervantes nach dem
frechen Plagiat, das an ihm begangen war, empfand. Die Darstellung
der Vorgänge in Barcelona ist matt und farblos. Don Quijotes Narrheit
tritt wenig in Aktion. Im ganzen scheint er weniger verrückt als die
Leute albern, die sich über ihn lustig machen wollen. Erst von da ab
nimmt die Erzählung wieder einen flotteren Gang an, wo der deus ex
machina Sanson Carrasco der Laufbahn Don Quijotes als fahrender
Ritter ein Ziel setzt. Der ursprüngliche Plan kommt wieder zur Geltung,
und die Handlung steuert schnell und folgerichtig dem Ende zu.

Cervantes will uns zwar glauben machen, dass er das Buch seines
Rivalen nicht gelesen habe, einige vielleicht unbeabsichtigte Reminiscenzen
daraus aber beweisen das Gegenteil. Man vergleiche folgende Stellen:

\begin{multicols}{2}
\centerline{Avellaneda}

{\itquoted\spanish\dots nos están aguardando
con una muy gentil olla de vaca, tocino, carneros, nabos y berzas,
que está diciendo: cómeme, cómeme.} (Cap.~IV.)

Nachdem Sancho einen Kapaun
und ein Dutzend Würstchen ({\it\spanish albondiguillas\/})
gegessen hat, verzehrt er noch: {\itquoted\spanish cuatro pellas de
manjar blanco\dots las otras dos que dél le quedaban se las metió
en elseno con intención de guardarlas para la mañana.} (Cap.~XII.)

Don Carlos fragt Sancho, ob
er tanzen kann: {\itquoted\spanish Pardiobre, señor,
que voltearía yo lindisamente,
recostado ahora sobre dos o tres
jalmas.} (Cap.~XII.)

\columnbreak
\centerline{Cervantes}

{\itquoted\spanish Señor huesped, dijo el ventero, lo que real y verdaderamente
tengo son dos uñas de vaca que parecen manos de ternera, ó
dos manos de ternera que parecen uñas de vaca: están cocidas con
sus garbanzos, cebollas y tocino, y la hora de ahora están diciendo;
cómeme, cómeme.} (II,~59.)

{\itquoted\spanish Acá tenemos noticia, buen Sancho, que sois tan amigo de
manjar blanco y de albondiguillas que si os sabran las guardáis en el
seno para el otro día.} (II,~62.)

Bei dem Ball zu Barcelona spricht Sancho über das Tanzen:
\dots {\itquoted\spanish zapateo como un girifalte:
pero en lo del danzar no doy puntada.} (II,~62.)
\end{multicols}

\originalpage{20}
Die drei {\it tocadores,} die Sancho von den Räubern gestohlen werden,
erinnern an die drei Dutzend {\it agujetas,} die ein Pikaro dem Knappen in
Zaragoza wegnimmt. (II,~67 und Avellaneda Cap.~XI.)

Die Ähnlichkeiten, die sich auch in den Kapiteln 1--58 des zweiten
Teiles finden, haben zu mancherlei Vermutungen Anlass gegeben. Dass
Cervantes hier Avellaneda nachgeahmt hat, scheint mir ausgeschlossen.
Andererseits würde eine Kenntnis des originellen zweiten Teiles bei
Avellaneda deutlichere Spuren zurückgelassen haben. Die Ähnlichkeiten
dieser Art sind kurz folgende: Der Zweikampf zwischen den Knappen
Sancho und Tomé Cecial (II,~14) erinnert an den geplanten Zweikampf
Sanchos mit dem schwarzen Knappen (Avellaneda Cap.~XXXIII), das
Verhalten Don Quijotes beim Puppenspiel (II,~26) an eine entsprechende
Scene bei Avellaneda bei einer Theateraufführung (Avellaneda Cap.~XXVII),
die Vorgänge an dem Hofe des Herzogs an die im Hause des Archipampano.
Irgend welche wörtliche Übereinstimmungen habe ich nicht
bemerkt. Es bleibt schliesslich nichts weiter übrig, als die citierten
Ähnlichkeiten aus der Gemeinsamkeit des Stoffes zu erklären\pfn{1}{\pxxni}.

\section{4. Das Werk Avellanedas.}

Viel auffallender als die behandelten Anklänge an Avellaneda bei
Cervantes sind nätürlich die Beziehungen zwischen dem Pseudo-{\it Don Quijote}
und dem ersten Teil. Vor allzu wörtlichen Übereinstimmungen,
hat sich der Verfasser wohl gehütet, er lebt ja in einer literarisch
interessierten Zeit, die ein derartiges Plagiat übel aufgenommen hätte.
Mit Vorliebe wählt Avellaneda Situationen von derber Komik zur Nachahmung
aus. Alles, was der erste Teil an Schmutz enthält, hat er sorgsam
zusammengekratzt und durch die obscönen Erzeugnisse seiner
eignen Einbildungskraft vermehrt.

Die Exposition der Fortsetzung ist durch den Schluss des ersten
Teiles gegeben. Don Quijote ist durch die gute Pflege der Haushälterin
und seiner Nichte wieder ganz vernünftig geworden. Er liest in
Erbauungsbüchern ({\it Flos Sanctorum} von Villegas, der {\it\spanish Guia de pecadores}
von Fray Luis de~Granada und den {\quoted Evangelien und Episteln für das
ganze Jahr in Vulgärsprache}) und besucht fleissig die Messe. Da erweckt
Sancho durch die Erwähnung eines Ritterbuches die alte Narrheit in
\originalpage{21}
ihm, die zum vollen Ausbruch kommt, nachdem einige Edle auf
dem Wege zu den Turnieren von Zaragoza in dem Dorfe die Nacht
verbracht haben. Einer von ihnen, Don Alvaro Tarfé, lässt bei Don
Quijote eine Mailändische Rüstung zurück, die dieser sich sofort als
ein Geschenk des weisen Alquife aneignet. Eines schönen Morgens
verlässt er mit Sancho das Dorf, um sich ebenfalls zu den Turnieren
von Zaragoza zu begeben. In der nächsten Herberge\pfn{1}{\pxxini} vollzieht sich
alles nach dem Muster des ersten Teiles: Das Wirtshaus hat seine
Maritornes\pfn{2}{\pxxinii}, die Don Quijote für eine gefangene Prinzessin hält und
wieder in ihr Reich\pfn{3}{\pxxiniii} einzetzen will, Don Quijote weigert sich beim
Fortgehen zu zahlen\pfn{4}{\pxxiniv}. Wörtliche Anklänge fehlen nicht, z.~B.:

\begin{multicols}{2}
Cap.~V. {\itquoted\spanish Señor caballero, aqui
no habemos menester cosa alguna,
salvo que vuesa merced ó este
labrador que consigo trae me
paguen la cena, cama, paja y
cebada, y vayanse trasesto muy
en hora buena}.
\columnbreak
I,~17: {\itquoted\spanish Solo he menester que
v.~m.\ me pague el gasto que esta
noche ha hecho en la venta así
de la paja y cebada de sus dos
bestias como de la cena y camas
etc.}
\end{multicols}

Nach sechs Tagen kommen die beiden Fahrenden nach Ariza,
wo Don Quijote eine neue Devise auf seinen Schild malen lässt. Denn
da er auf einen Brief\pfn{5}{\pxxinv}, den er an Dulcinea geschrieben hat, eine
ungnädige Antwort bekommen hat, beschliesst er, die Liebe zu ihr
aufzugeben und sich {\it\spanish Caballero desamorado} zu nennen. Er besteht ein
Abenteuer mit einem Melonengärtner, den er bald Orlando furioso\pfn{6}{\pxxinvi},
bald Aglante, bald Bellido de~Olfos nennt. Der Ritter und Sancho
werden verprügelt, und ihnen ihre Reittiere genommen\pfn{7}{\pxxinvii}. Der Pfarrer
von~Ateca verpflegt sie acht Tage und macht beim Abschied Don
Quijote etwa dieselben Vorstellungen über die Lügenhaftigkeit der
Ritterbücher, wie der Kanonikus von Toledo (I,~48). Don Quijote geht
\originalpage{22}
nach Zaragoza, wo er einen Gefangenen zu befreien sucht nach Muster
der Befreiung der Galeerensklaven (I,~22). Er greift dabei einen
Schreiber an, der sich vor dem Lanzenstoss auf dieselbe Weise rettet
wie der Barbier (I,~21), nämlich, indem er rücklings von dem Esel
heruntergleitet. Der Ritter wird durch das Eingreifen Don Alvaro Tarfés
vor der drohenden Geisselung bewahrt und von ihm in sein Haus
geführt\pfn{1}{\pxxiini}. Von hier ab begibt sich der Verfasser auf den Boden
eigner Erfindung. Da das Turnier bereits vorüber ist, nimmt Don Quijote an
einem Ringstechen teil: {\itquoted\spanish no es cosa nueva en semejantes regocijos
sacar los caballeros á la plaza locos vestidos y aderezados y con humos
en la cabeza de que han de hacer suerte, tornear, justar y llevarse
premios, como se ha visto algunas veces en ciudades principales y en
la misma Zaragoza.} Im Hause des Schiedsrichters Don Carlos, wo
man zu abend isst, empfängt Don Quijote die Herausforderung Bramidans
de Tajayunque (Ambossspalter), Königs von Cypern, eines Karnevalriesen,
der auf den Schultern von des Gastgebers Sekretär ruht. In
der folgenden Nacht träumt der Ritter, Bramidan sei ins Haus gedrungen
und trachte der ganzen Bewohnerschaft nach dem Leben.
In seinem Wahn prügelt er Sancho und die Dienerschaft\pfn{2}{\pxxiinii}, bis Don
Alvaro ihn beruhigt. Der Sekretär erscheint als Knappe des Riesen
und bestellt Don Quijote zum Zweikampf nach Madrid, wohin Don
Alvaro und Don Carlos wegen der Verheiratung von des letzteren
Schwester ebenfalls gehen wollen. Unterwegs nach Madrid macht Don
Quijote die Bekanntschaft eines Soldaten und eines Eremiten, mit denen
er in Ateca bei Mosén Valentin einkehrt.

Am nächsten Tage macht man wegen der grossen Hitze an einer
Quelle längere Rast und vertreibt sich die Zeit mit Geschichtenerzählen.
Der Soldat gibt die Novelle {\it\spanish El Rico desesperado} zum besten, der
Eremit den {\it\spanish Cuento de los Felices Amantes,} Sancho eine Erzählung
\originalpage{23}
--- {\it\spanish Cuento de nunca acabar}\pfn{1}{\pxxiiini} ---, die identisch mit
der I,~20 berichteten ist, nur dass es hier Gänse, dort Hämmel sind, die den Fluss zu
überschreiten haben. Das nächste Abenteuer ist eine plumpe Travestie der
Doroteaepisode. In einem Walde findet man heulend, an einen Baum gebunden,
ein schmutziges, hässliches Weib von etwa fünfzig Jahren: {\it\spanish Barbara
la de la cuchillada,} eine Garköchin aus Alcalá de Henares. Sie ist
von einem Studenten, der ihr die Heirat versprochen hatte, in den
Wald gelockt und ausgeraubt worden.

Don Quijote nennt sie Cenobia, Königin der Amazonen, und stellt
sich in ihren Dienst. Ihre Hässlichkeit stört ihn nicht. Sie ist verzaubert.
Dies unappetitliche Weibsbild --- ein schöner Ersatz für Dulcinea ---
begleitet den Ritter auf seinen weiteren Fahrten und befleckt
seine idealen Bestrebungen durch ihre schmutzige Gegenwart. Es ist
dieser Teil und alles bis zum Ende folgende entschieden das Schlechteste
an dem Buche, langweilig durch die ewige Wiederkehr derselben Motive,
Prügeleien und Verhöhnung des Ritters, und abstossend durch die
schlüpfrigen Situationen, die die Anwesenheit der Barbara, die im Grunde
eine geriebene {\it\spanish alcahueta} ist, mit sich bringt.

Don Quijote hält verrückte Ansprachen, lässt Kartelle anheften,
Sancho wird eingesperrt, von einigen Pikaros im Gefängnis gefoppt und
wieder freigelassen. Überall erregen die beiden und ihre Begleiterin die
Heiterkeit des Publikums. Auf dem Wege nach Alcalá trifft Don Quijote
zwei Studenten, die ihm Rätsel aufgeben. In einer Herberge vor Alcalá
stört er die Aufführung von Lope de~Vegas Komödie {\it\spanish El testimonio
vengado.}

Zum Scherz fordert ihn ein Schauspieler (als Don Carlos) zum
Zweikampf in Madrid heraus und gibt ihm den Schwanzriemen eines
Maultiers als Pfand, worüber sich ein Streit zwischen Sancho und dem
Besitzer entspinnt\pfn{2}{\pxxiiinii}. Bei der Ankunft in Madrid erregt Don Quijote
die Aufmerksamkeit eines hohen Würdenträgers und wird von diesem
in sein Haus geladen, wo er Streitigkeiten mit einem Pagen und einem
Polizisten hat. Nachdem man sich genügend über das Kleeblatt lustig
gemacht hat, werden sie in das Haus eines andern Würdenträgers geschafft,
der sich {\it Archipampano} nennt. Der Sekräter Don Carlos', der
\originalpage{24}
inzwischen ebenso wie Don Alvaro eingetroffen ist, erscheint wieder als
Riese und entpuppt sich als Infantin Burlerina, die Don Quijote in dieser
Verzauberung gefolgt ist, um ihn nach Toledo zur Hilfe gegen den
Prinzen von Cordoba zu rufen. Dadurch wollen Don Alvaro und Don
Carlos den Ritter nach Toledo ziehen, um ihn dort im Tollhause {\it\spanish (casa
del Nuncio)} unterzubringen. Barbara wird in ein Altweiberspital {\it\spanish (casa
de arrepentidas)} gehen. Sancho soll nebst seiner Frau, die er durch
einen Brief herbeiruft, bei dem Archipampano bleiben. Der Verfasser
nimmt Abschied von ihm, indem er noch eine besondere Geschichte
über seine weiteren Schicksale verspricht.
Don Quijote findet im Tollhaus zu Toledo Aufnahme. Zum Schluss
wird noch berichtet, dass er als geheilt entlassen von neuem seine Wanderungen
als fahrender Ritter in Begleitung eines Knappen, von dem es
sich später herausgestellt habe, dass es ein Weib war, unternommen
und auf diese Weise Alt-Kastilien durchschweift habe, {\itquoted\spanish llamándose el
Caballero de los Trabajos, los cuales no faltará mejor pluma que los
celebre}\pfn{1}{\pxxivni}.

Wenn wir das Fazit aus dieser Analyse ziehen, so sehen wir, dass
der Verfasser bemüht gewesen ist, den ersten Teil möglichst im gleichen
Sinne fortzusetzen. Es wäre aber wirklich ein Wunder, wenn ein Fortsetzer
so weit in die Konzeption des ersten Verfassers eingedrungen
wäre und sich so mit dessen Geiste identifiziert hätte, dass man sein
Werk dem Vorbilde gleichstellen könnte. Stets werden die Gestalten
des Originalwerkes in den Händen des Nachahmers etwas von ihrer
ursprünglichen Frische und Kraft verlieren. Sie werden entweder
schwächer und entfärbt oder überladen und karikiert sein. Avellaneda
hat die Eigenheit zu übertreiben, indem er einzelne Züge, die ihm
als besonders charakteristisch an seinem Vorbild erscheinen, gewaltsam
hervorhebt.

\blankline

Für seine Don Quijote-Gestalt musste Avellaneda die ursprüngliche
Idee beibehalten. Don Quijote ist der {\quoted überspannte Leser}, der die
Welt, wie sie seine Bücher schildern, für wirklich oder wenigstens für
verwirklichungsfähig hält. Sonst hat sich manches in {\emph Don Quijotes
Charakter} geändert. Geradezu unangenehm wirkt bei Avellaneda
die bis zur Geckenhaftigkeit ausgeartete Eingebildetheit des Ritters.
Bald nennt er sich Cid, bald Fernan Gonzalez, bald Achilles oder
\originalpage{25}
Ferdinand von~Aragon. Dieser Zug ist umso auffallender, als der
Verfasser es versäumt hat, eine andere Seite zu schildern, die die
unsympathischen Eigenschaften in Don Quijotes Charakter zu mildern
im stande ist: das Gemüt. Denn wir können nur so lange wahres
Interesse für seinen Helden haben, als wir auch wirkliche Sympathie für
ihn empfinden und seine Narrheit mit Mitleid ansehen.

Schon Cervantes hatte gerügt, dass Avellaneda seinen Ritter die
Liebe zu Dulcinea aufgeben lasse. Er, der sich den treusten aller
fahrenden Ritter, Amadis de Gaula, zum Muster genommen, musste
seiner Dame treu bleiben. Sein phantastisches Liebesleben war sein
ganzes Glück, sein Halt, seine Poesie. Cervantes drückt diesen Gedanken
zweimal\pfn{1}{\pxxvni} sehr schön aus: {\itquoted\spanish El caballero
andante sin dama es como el árbol sin hojas, el edificio sin cimiento,
y la sombra sin cuerpo de quien se cause.}

Avellaneda ist bemüht gewesen auch die pathologische Seite in
derselben Weise zu behandeln wie Cervantes. Hier wie dort ist Don
Quijote ein {\it\spanish loco entreverado.} Trotzdem zeigen sich noch gewisse
Unterschiede. Da wir in diesem Punkte mit rein aesthetischen Betrachtungen
nicht weit kommen würden, möchte ich zunächst mit Hilfe
von psychologischen Kriterien die Symptome von {\emph Don Quijotes
Krankheit} feststellen und klassifizieren.

Die Ursachen seines Übels lassen sich leicht erkennen. Er ist
von Natur schwächlich gebaut, schlecht genährt und neigt zur Schwärmerei.
Tag und Nacht liest er Ritterbücher, sodass er schliesslich vom vielen
Lesen und wenigen Schlafen verrückt wird. Sein Hirn trocknet ein,
sagt Cervantes, was ein moderner Arzt etwas gelehrter vielleicht mit
Gehirnatrophie bezeichnen würde. Die Idee des armen Junkers, das
fahrende Rittertum durch sein Beispiel wieder zu beleben, ist eine Art
Grössenwahn.

Die Symptome seiner Krankheit bestehen in Sinnestäuschungen.
Meistens ist er Apperzeptionsillusionen unterworfen, d.~h.\ solchen
Trugwahrnehmungen, deren Reizquelle in wirklichen Sinneseindrücken besteht,
die durch subjektive, aus dem eignen Vorstellungskreise stammende
Elemente verfälscht werden. Die Undeutlichkeit resp.\ Mehrdeutigkeit
der Wahrnehmung begünstigt das Entstehen von Illusionen. Bei
kranken oder auch nur psychisch geschwächten Individuen nehmen sie
\originalpage{26}
die Schärfe wirklicher Sinneseindrücke\pfn{1}{\pxxvini} an. Denn da sind die
Bedingungen für die Entstehung von Auffassungverfälschungen ausserordentlich
günstig: starke, gemütliche Erregungen, grosse Lebhaftigkeit
der Vorstellungen und endlich Unfähigkeit zu einer verständigen
Sichtung und Berichtigung des Erfahrungsmaterials\pfn{2}{\pxxvinii}. Scharf zu scheiden
von den peripher bedingten Trugwahrnehmungen sind die Hallucinationen,
d.~h.\ solche Trugwahrnehmungen, bei denen eine äussere Reizquelle
gar nicht vorhanden ist. Eine einfache Reminiscenz genügt, Phantasmen
zu erzeugen, die wirklichen Sinneseindrücken gleichwertig sind\pfn{3}{\pxxviniii}.
Trugwahrnehmungen der zweiten Art finden wir nur selten im {\it Don Quijote.}

Um speziell für Don Quijote die einzelnen Phasen eines Illusionsvorganges
von der Entstehung bis zur Wiederauflösung abteilen zu
können, bediene ich mich des Abenteuers mit den Hämmeln\pfn{4}{\pxxviniv} als
Musterbeispiels. Da die einzelnen Momente, wie ich zu gleicher Zeit
zeigen werde, in den übrigen Abenteuern wiederkehren, können sie als
typisch für den Verlauf eines Anfalles gelten.

1. Das {\emph illusionserregende Moment}: in unserm Falle eine
Staubwolke, die von der nahenden Hammelherde aufgewirbelt wird
Das~Barbierbecken auf dem Kopfe des Barbiers leuchtet in der Sonne wie
Gold. Bei dem Abenteuer mit Maritornes ist es Nacht. Das Lärmen
der unsichtbaren Walkmühle, die langen Arme der Windmühle etc.

2. Die {\emph Reminiscenz}. Durch das Wahrgenommene werden Erinnerungen
aus der Lektüre ausgelöst. Eine nahende Staubwolke
deutet in den Ritterbüchern gewöhnlich ein herankommendes Heer an,
Prinzessinnen suchen nächtlicherweile die fahrenden Ritter auf etc.

3. Das aus der Mischung des Apperzipierten und des Erinnerten
hervorgehende {\emph Urteil}: Es ist ein nahendes Heer, eine Prinzessin,
Riesen u.~s.~w.

4. {\emph Totale Illusion}. Die in der Illusion bestehende Sachlage
wirkt bestimmend auf das Handeln des Ritters. Wie man aus dem
Verhalten Sanchos sieht, wäre jetzt eine Korrektion der Täuschung
möglich. Die Illusion ist jedoch zu einem so hohen Grade und zu
solcher Deutlichkeit vorgeschritten, dass die Erkenntnis des wahren
Sachverhaltes nicht eintreten kann. Das ist das eigentliche pathologische
\originalpage{27}
Symptom. Don Quijote reitet mit gefällter Lanze auf die Hammelherde
ein, greift die Windmühle an etc.

5. Die {\emph Auflösung der Illusion} erfolgt gewöhnlich auf gewaltsamem
Wege. Don Quijote wird verprügelt, fällt vom Pferde u.~s.~w.

6. Die {\emph Hilfsillusion}\pfn{1}{\pxxviini}. Den schlechten Ausgang des Abenteuers
erklärt er im Sinne seiner Ritterbücher. Sein Pferd ist schuld an dem
Unglück, oder es sind böse Zauberer im Spiel. Es ist dies eine
Aushilfe, die direkt aus der konsequenten Durchführung der Wahnidee folgt.

Avellaneda gelingt es nicht immer, den ganzen psychischen Prozess
mit derselben Folgerichtigkeit vorzuführen wie Cervantes. Wenn Don
Quijote eine Schenke für ein Schloss hält, im nächsten Augenblick aber
von einem {\itquoted\spanish ventero andante} spricht (Cap.~IV.)\ oder sagt:
{\itquoted\spanish ¡Entramos en la venta!}~(Cap.~XXVI),
so liegt hier entschieden ein Missgriff des
Verfassers vor. Aller Wahrscheinlichkeit und psychologischen Begründung
entbehrt Don Quijotes Verhalten bei dem Ringstechen zu Zaragoza.
Wirklich durchgeführt ist der Illusionsvorgang in dem Abenteuer mit
dem Melonenwächter und dem mit Barbara. In ersterem wirkt die
Lanze illusionserregend, in letzterem ein aus dem Gebüsch dringendes
Geschrei\pfn{2}{\pxxviinii}. In beiden Fällen finden wir auch die charakteristischen
Hilfsillusionen. Das Abenteuer mit dem Riesen Tajayunque steht auf
einer Stufe mit der Micomiconaepisode (I,~29), da es sich darin um
künstlich zurecht gemachte Situationen handelt, die bei Don Quijote
nur den Glauben an Riesen und hilfesuchende, vertriebene Prinzessinnen
voraussetzen. Träume als illusionserregendes Moment sind verwendet
in Don Quijotes nächtlichem Kampf mit dem vermeintlichen Riesen
(Av.\ Cap.~XIII) und seinem Angriff auf die Weinschläuche (I,~36).
Wirkliche Tobsuchtsanfälle von hallucinativer Natur, wie sie bei
Avellaneda in Cap.~III und~X vorkommen, können nicht als eine Bereicherung
des Don Quijotemotives gelten. Denn sie erregen eher
Unlust als Heiterkeit. Der Junker tritt darin zu sehr als Irrsinniger
hervor. Ein derartiger Geisteszustand musste unbedingt ins Tollhaus führen.

Anders hat Cervantes in seinem zweiten Teil das Problem gefasst.
Er schildert uns Don Quijote in einem neuen Stadium seiner Krankheit.
\originalpage{28}
Die Symptome der beginnenden Heilung treten immer zahlreicher auf,
die Rückfälle werden seltner, sodass uns die Gesundung am Ende des
Buches nicht unerwartet kommt. Die einzelnen Abenteuer sind sämtlich
der Art, dass sie bei dem Ritter nur den Glauben an die Möglichkeit
des fahrenden Rittertums und an die Existenz fabelhafter Wesen
voraussetzen. Das Hallucinative tritt vollkommen in den Hintergrund. Der
Ritter sieht die Wirtshäuser nicht mehr als Schlösser an, wie an zwei
Stellen besonders hervorgehoben wird. Die Abenteuer treten in einer
solchen Gestalt an ihn heran, dass sie einer Umbildung durch die
Phantasie nicht mehr bedürfen. Die Handlungsweise Don Quijotes wird
allerdings durch die Reminiscenzen aus den Ritterromanen bestimmt.
Unter diese Kategorie fallen: der Kampf mit dem Spiegelritter, die
Begegnung mit den Darstellern des {\it\spanish Auto de las Cortes de la Muerte,}
das Bravourstückchen mit dem Löwen, schliesslich alle von dem Herzogspaar
inscenierten Abenteuer. Hilfsillusionen sind in den meisten Fällen
entbehrlich. Bei dem Puppenspiel ist Don Quijote nur jener theatralischen
Illusion\pfn{1}{\pxxviiini} unterworfen, die manchmal auch bei normalen Menschen eintritt.
Wesentlich ist dabei auch, dass er nicht in üblicher Weise einen Ausweg
sucht, sondern seinen Irrtum anerkennt und den Marionettenspieler
entschädigt. Das Erlebnis in der Höhle des Montesinos ist ein Traum,
fällt also nicht in die Klasse der Sinnestäuschungen. Don Quijote
selbst steigen später Zweifel an der Realität des Geträumten auf.

Jedenfalls ist der Junker im echten zweiten Teil auf eine ganz neue
Art aufgefasst worden. Trotzdem kann ich diesen Band in ästhetischer
Hinsicht nicht auf eine Stufe mit dem ersten Teil stellen. Goethe scheint
mir vollkommen das Richtige getroffen zu haben, wenn er sagt: {\quoted Solange
sich der Held Illusionen macht, ist er romantisch, sobald er bloss gefoppt
und mystifiziert wird, hört das wahre Interesse auf}\pfn{2}{\pxxviiinii}.

Dem Fortsetzer könnte man allerdings den Vorwurf machen, dass
er den übernommenen Typus nicht erfolgreich variiert habe.

\originalpage{29}
Der {\emph Sancho} Avellanedas, der die Bewunderung Lesages erregte,
bringt gegenüber seinem Vorbild nicht viel Neues. Auch an ihm vermisse
ich die sympathischen Züge: seine Gutherzigkeit und seine Treue.

Von den bei Avellaneda neu hinzugetretenen Personen ist nur eine
von Wichtigkeit. Der Dorothea des ersten Teiles entspricht in Avellanedas
{\it Don Quijote} die {\emph Barbara}, ein altes hässliches Weib , der der
Beruf einer Kaldaunenköchin als Vorwand für alle möglichen unehrlichen
Gewerbe dient. Die Verwandtschaft mit der Celestina ist offenbar. Es
ist die schlaue Kupplerin, die aus dem mittellateinischen {\it Pamphilus de
amore} stammend durch die geniale Behandlung, die sie von Juan Ruiz
als Trotaconventos und von Fernando de~Rojas als Celestina erfahren
hat, in der spanischen Litteratur Heimatsrecht erworben hat. Wenn von
Barbara berichtet wird, dass sie in Alcalá wegen einer Narbe auf der
Wange unter dem Spitznamen {\itquoted\spanish la de la cuchillada} bekannt war, so
erfahren wir das Gleiche von ihrer berühmten Vorgängerin: {\itquoted\spanish aquella
vieja de la cuchillada que solía biuir en las lenerías á la cuesta del río}\pfn{1}{\pxxixni}.
Das {\it persignum crucis} hat sein Gegenstück in dem {\itquoted\spanish fermosa
con aquel su Dios os salve que traviessa la media cara}\pfn{2}{\pxxixnii}.
Auch dieselben Fertigkeiten werden Barbara nachgerühmt wie der Celestina. In
einem unterscheiden sich beide. Barbara ist niedrig und gemein. Es fehlen ihr jene
Genialität im Laster, die der Celestina etwas Dämonisches gibt, und ihre
Kenntnis der Schwächen der menschlichen Seele, die sie zur Beherrscherin
der Situation macht. Dass Avellaneda ihr den Namen der Kaiserin von
Palmyra Cenobia\pfn{3}{\pxxixniii} gibt und sie zur Amazonenkönigin macht, scheint mir
in keinem Ritterbuch begründet zu sein. Im {\it Orlando Furioso} ist sie
allerdings schon einmal erwähnt worden\pfn{4}{\pxxixniv}. Auch die Gabrina\pfn{5}{\pxxixnv} dieses
Gedichtes hat einige Ähnlichkeit mit Barbara.

Einen ästhetischen Vorteil hat die Einführung dieser Gestalt, wie
wir bereits in der Analyse sahen, nicht gebracht.

\blankline

Wie er in der Vorrede erklärt, ist es auch Avellanedas Absicht,
jene verderblichen Ritterbücher aus der Welt zu schaffen. Cervantes
satirisiert die Ritterromane, indem er einen Junker darstellt, der die
\originalpage{30}
Ritterromane für historisch hält und ihren Inhalt realisieren will, da er
meint, dass seine Zeit gegenüber der Welt jener Bücher einen verderbten
Zustand darstelle. Die Unsinnigkeit dieses Unternehmens, das fortwährend
in seinen Berührungen mit dem Alltagsleben scheitert, soll die ganze
Unwahrheit der Ritterbücher offenbaren. Die einzelnen Abenteuer sind
Parodien auf ähnliche Situationen in den verspotteten Romanen. Alles
ist in eine niedere Sphäre gerückt. Es sind nicht Ritter, gegen die der
Junker kämpft, sondern Maultiertreiber, nicht Riesen, sondern Windmühlen,
nicht ganze Heere, sondern Schafherden. Eine {\emph Satire,} die sich solcher
Mittel bedient, ist wohl eher burlesk als grotesk zu nennen, wie
Schneegans\pfn{1}{\pxxxni} tut, wenn wir nämlich als wesentliches Merkmal des Grotesken
die masslose Übertreibung, des Burlesken aber die Mischung des Erhabenen
mit dem Trivialen ansehen.

Zur weiteren Begrenzung der von Cervantes ausgeübten Satire kann
auch ein Vergleich mit dem {\it Rasenden Roland} dienen.

{\quoted Das Mittel der komischen Darstellung liegt bei Ariosto in der
Natürlichkeit und Realität seiner Darstellung, wenn er sie auch für das
Wunderbarste und Phantastischste in Anwendung bringt}\pfn{2}{\pxxxnii}. Ihm kommt
es gar nicht darauf an, gegen die Ritterlitteratur zu polemisieren. Er
freut sich an den fabelhaften Erfindungen der Spielmannsepen und trägt
sie mit der Skepsis des Renaissancemenschen vor. Cervantes dagegen
will nicht so sehr die ästhetische Minderwertigkeit der Ritterbücher als
die Schädlichkeit betonen, die sie mit der einseitigen Anregung des
Phantasielebens in sich bergen. Dieser lehrhafte Zug ist das wesentlichste
Merkmal, das die satirische Anschauungsweise von der ironischen trennt.

Die Satire misst ihr Objekt an etwas Idealem, Seinsollendem, sodass
die durch sie getroffenen Zustände klein, kümmerlich, ja schädlich
erscheinen. So verwandt auch das Satirische und das Komische darin
sind, dass beide darauf ausgehen, ihr Objekt ad absurdum zu führen,
darf man sie doch nicht ohne weiteres identifizieren. Die Satire bedient
sich bei der Verfolgung ihrer lehrhaften, ethischen oder frivolen Absichten
der Komik nur als Ausdrucksmittels.

Dem Humor fällt neben der Satire noch eine besondere Aufgabe
zu. Er soll ihre zersetzende Tendenz durch seinen optimistischen
versöhnenden Gehalt ausgleichen.

Mittel und Ziel der Satire waren für Avellaneda gegebene Faktoren.
Es fragt sich nur, ob er ebenso gut trifft wie Cervantes. Es
\originalpage{31}
kommen natürlich nur die von ihm selbst erfundenen Situationen
in Betracht.

Sein Ritter durchzieht alle möglichen Dörfer und Städte, heftet
Kartelle an und fordert die Edelleute des Ortes zum Kampfe heraus,
obgleich dies keineswegs die Haupttätigkeit der Romanhelden ist. Mehr
im Sinne der Ritterbücher ist es, wenn Avellaneda seinen Don Quijote
durch ein aus dem Gebüsch dringendes Geschrei in ein Abenteuer verwickelt
oder ihn durch einen Riesen zum Kampf herausfordern lässt.
Das Ringstechen in Zaragoza zeigt eher, wie die blutigen Turniere der
alten Zeit zu einem blossen Spiel geworden waren, als dass es eine Satire
auf die in den Romanen geschilderten Kampfspiele wäre.

Auch nach der formalen Seite hat Cervantes die Ritterbücher lächerlich
gemacht, indem er den verschrobenen Stil nachahmt, der den meisten
Romanen dieser Art, abgesehen von den in mustergültiger Prosa geschriebenen
ersten Büchern des {\it Amadis}, eigen ist. Dabei hat er nicht
übertrieben. Die komische Wirkung geht nur aus dem Nebeneinander
von zwei verschiedenen Stilarten hervor. Das grösste chevalereske Kauderwelsch
Don Quijotes reicht noch nicht an manche der Stilkunststücke
eines Feliciano de~Silva heran. Da das Karikierende fehlt, wird man dies
Verfahren eher der Ironie zuzählen, wie überhaupt bei Cervantes ironischer
und satirischer Humor ohne scharfe Trennung durcheinandergehen.

Avellaneda hat sich nicht ohne Geschick desselben Mittels bedient,
übrigens nur da, wo Don Quijote spricht. Sonst fällt er aus dem einfachen
Erzählerton nicht heraus. Von jenen präludierenden Eingängen\pfn{1}{\pxxxini},
die Cervantes nach dem Muster der Ritterromane (namentlich des
{\it Caballero del Febro\/}) als Einleitung zu einzelnen Abenteuern und
Kapiteln dienen lässt und die meist eine schwülstige Schilderung des
Sonnenaufganges geben, haben wir nur eine schwache Kopie bei
Avellaneda: {\itquoted\spanish Tres horas ántes que el rojo Apolo esparciese sus
rayos sobre la tierra, salieron de su lugar} (Cap.~IV). Sonst gehen die Kapitel
ohne merkliche Einleitung und Schluss ineinander über. Nirgends finden
sich persönliche Betrachtungen des Verfassers, Anmerkungen des
Chronisten oder eine direkte Wendung an den Leser. Mit dieser
Objektivität steht der falsche {\it Don Quijote} ganz vereinzelt in der
damaligen Romanlitteratur, wo überall die Persönlichkeit des Verfassers
in der Ichform oder in der Substitution eines Chronisten durchscheint.

\originalpage{32}
Die direkte Polemik gegen die Ritterbücher, von der Cervantes.
in dem {\itquoted\spanish Escrutinio} (I,~6) so geschickt Gebrauch macht, findet nirgends.
bei Avellaneda Anwendung. Er begnügt sich damit, die Ritterromane
{\quoted lügenhaft} und {\quoted verderblich} zu nennen.
Auf die Höhe einer durchgeführten sozialen Satire, die bei Cervantes
ganz beiläufig angebracht ist und doch dem Buch eine allgemeinere
Bedeutung erteilt\pfn{1}{\pxxxiini}, hat sich Avellaneda nicht erheben können. Einzelne
Bemerkungen satirischen Inhaltes fehlen jedoch nicht. Nach dem Muster
des pikaresken Romans hält er sich über die schlechte Beschaffenheit
der von den Wirten gelieferten Speisen (Cap.~V und~XXVII) und über
die Gewalttätigkeit der Studenten in Alcalá (Cap.~XXVIII) auf. Endlich
gehören noch die Auslassungen Sanchos über das Thema {\itquoted\spanish la vida de
palacio es vida bestial} (Cap.~XXXV) hierher.

Ich habe bereits angedeutet, dass ein prinzipieller Unterschied
zwischen Satire und Komik im engeren Sinne besteht. Während nämlich
die Komik dann entsteht, wenn zunächst getrennt verlaufende
Vorstellungsreihen in unserem Bewusstsein plötzlich und unerwartet in irgend
welche Beziehung zu einander gesetzt werden, handelt es sich bei der
Satire um einen Konflikt zwischen Objekten, die mit Anspruch auf
Achtung auftreten, und einem Ideal, das Eigentum des Dichters ist. Die
Zerstörung des scheinbar Wertvollen durch die Anschauungsweise des
Satirikers, die wir auf dem Wege der Einfühlung zu unserer eignen
machen, hat eine ähnliche Mischung von Lust- und Unlustempfindungen
zur Folge wie der gleiche Prozess in der Komik. Diese Wirkung wird
umso grösser sein, wenn die von dem Satiriker beigebrachten
Gesichtspunkte uns neu sind.

So kommt es, dass eine spätere Zeit einer Satire niemals volles
Verständnis entgegenbringen kann. Einerseits fehlt ihr Kenntnis und
Schätzung des satirisierten Gegenstandes, andererseits die Unbefangenheit
des zeitgenössischen Lesers. Der moderne Mensch wird die spezielle
Satire des {\it Don Quijote} kaum mit denselben widerstreitenden Gefühlen
aufnehmen wie der Zeitgenosse. Für ihn wird das Buch in erster Linie
ein Werk des Humors sein. Die rein ästhetische Wirkung kann bei
dieser Verschiebung in der Auffassung nur gewinnen.

Das Bestreben des Ritters, Dinge auszuführen, die weder seinen
Kräften noch der Sachlage angemessen sind, muss in seinem sicheren
\originalpage{33}
Misslingen eine Menge von rein komischen Situationen zeitigen\pfn{1}{\pxxxiiini}.
Anders verhält es sich, wenn die {\emph Komik} aus künstlich hergestellten
Situationen entspringt, deren Gelingen durch seine Narrheit bedingt ist.
Hierdurch entsteht eine besondere Art von Komik: {\quoted das Possenhafte}\pfn{2}{\pxxxiiinii}.
Es ist etwas Gewolltes, Gemachtes, etwas, das auf eine intellektuelle
Schwäche dessen spekuliert, den es in eine komische Beleuchtung rücken
will. Die Absicht des Gefoppten, etwas vorzustellen, was er nicht vorstellen
kann, erhöht die komische Wirkung. Wenn Cervantes in seinem
zweiten Teil dem Possenhaften den Vorzug gibt, so liegt der Grund
dafür in der psychologischen Entwicklung seines Helden\pfn{3}{\pxxxiiiniii}. Die komische
Litteratur vor Cervantes in Spanien --- Lustspiel, {\quoted entremeses}, pikaresker
Roman --- macht fast ausschliesslich Gebrauch vom Possenhaften. Die
Schlauheit, die über die Dummheit, Leichtgläubigkeit und Gutmütigkeit
triumphiert, bildet überall das Hauptthema.

Avellaneda arbeitet mit denselben Mitteln wie Cervantes. Die annähernd
folgerichtige Durchführung des Illusionsproblems musste im
Verein mit der parodistischen Nachahmung von Situationen aus den
Ritterbüchern, dem an die Satire gebundenen Teil der Komik, dieselben
komischen Effekte hervorbringen.

Die Komik des {\it Don Quijote} beruht im wesentlichen auf der
Selbsttäuschung\pfn{4}{\pxxxiiiniv}, in der der Held befangen ist. Und zwar ist es eine
Selbsttäuschung doppelter Natur: einmal die aus seinem Geisteszustand
resultierende Illusion und dann sein Irrtum, dass die ihm zu Gebote
stehenden Mittel für seine Zwecke ausreichen. Die zweite Art ist,
obgleich abhängig von der ersteren, da ja der Ritter erst kraft seiner
Illusion sich für stark und seine Waffen für gut hält, dennoch wichtig
für die komische Wirkung. Denn, wäre der Held kräftig und wohl
ausgerüstet, würde er in seiner Verblendung wohl eher tragisch als
komisch erscheinen. Zu der Komik der Selbsttäuschung gesellt sich
noch die Komik der äusseren Erscheinung. Wenn der moderne Mensch,
\originalpage{34}
der gewohnt ist, geistige Schäden auf eine Stufe mit körperlichen zu
stellen und daher für die Komik, die aus ihnen entstehen kann,
weniger Sinn hat als der naive Mensch, bei der Lektüre des {\it Don
Quijote} nicht Missfallen darüber empfindet, dass ein armer kranker
Mann zum Gegenstand des Gelächters gemacht wird, so liegt der
Grund darin, dass die in dem Missverhältnis von Mittel und Zweck
und in der äusseren Erscheinung hervortretende Komik ihn über
solche humane Regungen hinwegtäuscht. Wenn uns bei der Lektüre
von Avellanedas Buch viel eher ein Gefühl des Widerwillens gegenüber
der Narrheit des Ritters beschleicht, liegt das jedenfalls an einem
Zuwenig von Abwechslung zwischen Illusionskomik und Erscheinungskomik.
Es gelingt ihm selten, in uns ein lebhaftes Bild der äusseren
Erscheinung des Ritters {\quoted von der traurigen Gestalt} zu erzeugen. Viel
bedeutender als diese Differenz zwischen beiden Werken, dem des
Cervantes und dem Avellanedas, ist eine andere: Dem echten {\it Don
Quijote} liegt eine wahrhaft humoristische Weltanschauung zu Grunde.

Nach Eduard von Hartmann\pfn{1}{\pxxxivni} entsteht der {\emph Humor} aus einer
Kombination des Komischen mit dem Rührenden oder dem Tragischen
oder mit beiden zugleich. Darin liegt gerade die sittliche und
ästhetische Bedeutung des Humors, dass er höhere Werte mit in den
Strudel der komischen Vorstellungsbewegung hineinzieht, schliesslich aber
doch das sittlich Wertvolle hervortauchen und sich herauskristallisieren lässt.

Der Don Quijote des Cervantes ist wirklich humoristisch, weil ihn
der Verfasser mit edlen Eigenschaften ausgestattet hat, die sich neben
den komischen Seiten seines Charakters siegreich behaupten und ihm
unsere Sympathie sichern. Bei Avellaneda ist er nur komisch, denn
hier treten nur seine lächerlichen Seiten hervor. Am besten zeigt sich
der Unterschied der beiden Behandlungsarten in den Schlüssen. Wie
rührend komisch ist doch das Ende von Don Quijotes Ritterlaufbahn
bei Cervantes! Erst sein Schmerz über die unglückselige Niederlage,
dann die aufrichtige Freude über Don Gaspar Gregorios Rückkehr, sein
komischer Entschluss, ein Schäfer zu werden und schliesslich seine
Heilung und der Tod, durch den er, Alonso Quijada der~Gute, seine
Angehörigen und Freunde tiefbetrübt zurücklässt, --- das alles ist von
dem Romandichter so humoristisch und gefühlvoll dargestellt worden,
dass uns das Ende von Avellanedas Roman, wo Don Quijote in ein
Narrenhaus gesperrt wird, roh und geschmacklos erscheinen muss. Dort
\originalpage{35}
erhebt sich der Humor zu einem so hohen Mass von Pathos, dass er
nahe an das Tragische grenzt, hier sinkt die Komik zur Banalität herab.

Eine gesonderte Betrachtung verlangt in beiden Werken der
wackere Schildknappe Sancho. Er ist naiv humoristisch, er handelt
humoristisch seiner Natur, seiner Anlage nach. Der Sancho Avellanedas
ist entschieden seine beste Figur. Darin dürfen wir, wenn auch mit
Vorsicht, dem Urteile Lesages\pfn{1}{\pxxxvni} beistimmen. Das Derbkomische lag
Avellaneda am besten. Jedoch scheint mir sein Sancho nicht frei von
fremden Einflüssen zu sein. Sehen wir bei Cervantes an dem Knappen
noch reine Charakterkomik, so finden wir auf der anderen Seite, dass
Avellaneda einzelne Züge von komischer Wirkung, seine bäurische
Sprechweise, seinen grossen Appetit usw.\ in einer Weise übertreibt,
dass sie fast zum Selbstzweck werden. Sancho rückt damit in die
Sphäre der {\quoted lustigen Person}, insbesondere des Gracioso der spanischen
Komödie. Besonders deutlich zeigt sich diese Verwandtschaft mit dem
Gracioso in der Karikierung des Ritters durch den Diener\pfn{2}{\pxxxvnii}. Vor dem
{\quoted Archipampano} z.~B.\ ahmt der Knappe seinen Herren in Stellung und
Rede nach\pfn{3}{\pxxxvniii}. Beider Reden will ich wörtlich anführen, da sie mir
charakteristisch für Avellanedas Stil erscheinen.

Don Quijote: \dots {\itquoted\spanish quedando Don Quijote puesto en mitad de la
sala, mirando á todas partes con mucha gravedad, puesto el cuento de
la lanza, que un criado le trajo, en tierra, estuvo callando \dots ; y
cuando vió que callaban y estaban aguardando á que él hablase, con
voz serena y grave comenzó á decir: Magnánimo, poderoso y siempre
augusto archipámpano de las Indias, decendiente de los Heliogábalos,
Sardanápalos y demás emperadores antiguos: hoy ha venido á vuestra
real presencia el Caballero Desamorado, si nunca le oistes decir, el
cual, despues de haber andado la mayor parte de nuestro hemisferio, y
haber muerto y vencido en él un número infinito de jayanes y descomunales
gigantes, desencantando castillos, libertando doncellas} etc.

Sancho: {\itquoted\spanish Púsose Sancho luego en medio, y volviendo la cabeza,
dijo á Don Quijote: Déme vuesa merced esa lanza, para que me
ponga como vuesa merced estaba cuando hablaba al Arcapámpanos \dots
y poniendo las manos en arco sin quitarse la caperuza, con no poca
\originalpage{36}
risa de los que le miraban, estuvo un buen rato sin hablar, hasta que
viéndolos callar, comenzó á decir, procurando empezar como su amo
Don Quijote, a cuyas razones habia estado no poco atento: ¡Magnánimo,
poderoso y siempre agosto harto de pámpanos \dots! \dots Habrá vuesa
merced de saber, señor decendiente del emperador Eliogallos y Sarganápalos,
que yo me llamo Sancho Panza el escudero, marido de Mari-Gutierrez
por delante y por detrás, si nunca le oistes decir \dots y há dias
que ando en mi rucio con mi señor por la mayor parte de este nuestro
\dots Y volviendo la cabeza á su amo le dijo: ¿Como diablos se llama
aquel? ¡Oh maldito seas! replicó Don Quijote: hemisferio, simple.
¿Pues qué quiere agora? replicó Sancho \dots ¿piensa que el hombre
ha de tener tanta memoria como el misal? Digo pues, prosiguió Sancho,
que tornando a mi cuento, señor rey de Hemisferio, yo no he hasta
agora muerto ni dispilfarrado aquellos gigantones que mi amo dice} etc.

Auch die Feigheit Sanchos ist als charakteristische Eigenschaft des
Gracioso anzusehen, vgl.~sein Verhalten bei dem Abenteuer mit Barbara.

Die meist sehr faulen Witze mögen von jenen Spassmachern inspiriert
sein, die bei den Umzügen Don Quijote in der Maske des
Schildknappen begleiteten\pfn{1}{\pxxxvini}.

Ein komisches Element verdient noch besondere Erwähnung, weil
es dem {\it Don Quijote} von Cervantes fremd ist: {\emph das Obscöne}. Das
Obscöne bildet zwar zu allen Zeiten einen Litteraturzweig für sich,
bisweilen aber nimmt es derart überhand, dass es auch die echte Kunst
angreift. In eine solche Zeit fällt Avellanedas {\it Don Quijote.} Man
beachte nur die Wendung zum Pornographischen, die der pikareske Roman
mit der {\it Pícara Justina} von Andrés~Perez genommen hat. Auf einer
höheren künstlerischen Stufe können wir diese Erscheinung finden, wenn
wir den Übergang von Lope de~Vegas graziöser Leichtfertigkeit zu der
spielenden Frivolität Tirso de~Molinas beobachten, die ihn die gewagtesten
Scenen auf die Bühne bringen liess. Dazu stimmen auch die Klagen,
die verschiedentlich darüber erhoben wurden, dass das Volk an dem
auf der Bühne dargestellten Unmoralischen solchen Gefallen fände\pfn{2}{\pxxxvinii}.
Wir dürfen also einen Teil von Avellanedas Schuld auf Rechnung seiner
Zeit setzen. Die Zote, mag sie nun geschickt verhüllen oder ungeschickt
entblössen, ist stets gleich unmoralisch. Obgleich sich Sancho auch
nicht gerade gewählt ausdrückt, so ist es die Aufgabe der Barbara, den
Schmutz in die Don Quijote-Fabel zu tragen. Ein Beispiel mag genügen:

\originalpage{37}
{\itquoted\spanish Señor caballero, respondía ella {\rm (Barbara),} yo quisiere ser de quince
años y más hermosa que Lucrecia, para servir con todos mis bienes
habidos y por haber á vuesa merced; pero puede creer que si llegamos
á Alcalá, le tengo de servir allí, como lo verá por la obra, con un par
de truchas que no pasen de los catorce, lindas á mil maravillas y no
de mucha costa. Don Quijote, que no entendía la música de Bárbara,
le respondió. Señora mia, no soy hombre que se me dé demasiado
por el comer y beber: con eso á mi escudero Sancho Panza; con todo,
si esas truchas fueren empanadas, las pagaré, y las llevarémos en las
alforjas para el camino; aunque es verdad que mi escudero Sancho,
en picándosele el molino, no dejará trucha á vida}. (Cap.~XXIII.) So soll
durch das Anerbieten Barbaras die Tugendhaftigkeit des Ritters in ein
komisches Licht gerückt werden, wie an einigen weiteren Stellen Sanchos
Harmlosigkeit. Wie viel geistvoller ist die satirische Behandlung des
gleichen Motives bei Cervantes! Don Quijote, der sich nach dem Muster
seines Vorbildes Amadis stets seiner tugendhaften Treue rühmt\pfn{1}{\pxxxviini},
bildet sich ein, dass ihn verliebte Prinzessinnen nächtlicherweile besuchen.

Auch die stilistische Komik fällt meistens Sancho zu. Hier finden
wir dieselben Mittel der komischen Darstellung wie bei Cervantes:
bäuerliche Redeweise, Anführung von Sprichwörtern, Wortverdrehungen
(vgl.\ das auf voriger Seite citierte Beispiel) usw. Dazu kommt noch
Sanchos kameradschaftliches Verhältnis zu Rocinante und dem Esel und
das Hineinlegen von menschlichen Fähigkeiten in die Reittiere. Auch
hier ist Cervantes vorbildlich gewesen\pfn{2}{\pxxxviinii}.

Vereinzelt zeigen sich Ansätze zum {\emph grotesken Stil} z.~B.:
{\itquoted\spanish Juro por el orden de caballería que recebí, que solo por eso que has
dicho, y porque entiendas, que no puede caber temor alguno en mi
corazón, estoy por volver al lugar y desafiar á singular batalla, no
solamente al Cura, sino á cuantos curas, vicarios, sacristanes, canónigos,
arcedianos, deanes, chantres, racioneros y beneficiados tiene toda la
Iglesia romana, griega y latina, y á todos quantos barberos, médicos,
cirujanos y albeitares militan debajo la bandera de Esculapio, Galeno,
Hipócrates y Avicena.} (Cap.~IV.) Auch die Schilderungen von Sanchos
\originalpage{38}
ungeheurer Esslust erinnern an ähnliche in Rabelais {\it Gargantua und
Pantagruel\/} und Pulcis {\it Morgante maggiore.}\pfn{1}{\pxxxviiini}

Der komischen Ausdrucksweise sind vielleicht die bei Avellaneda
vorkommenden Vereinigungen von Dingen, die nicht zusammengehören,
zuzuzählen, wie: {\itquoted\spanish Aqui no hay castillo ni fortaleza, y si
alguna hay es la del vino} (S.~13\sup{b}) {\itquoted\spanish Con ellos
(ochenta ducados) y notable gusto
nos salimos una tarde de Alcalá} (S.~70\sup{a}). Ähnliches finden wir ja auch bei
anderen komischen Dichtern, bei Heinrich Heine und oft bei den Franzosen.

\blankline

In einer Darstellung der {\emph litterarischen Einflüsse,} die ausser
Cervante's erstem Teil bei der Entstehung des falschen {\it Don Quijote}
mitgewirkt haben, würde ich kaum annähernde Vollständigkeit erreichen,
da die meisten Ritterbücher zu bibliographischen Seltenheiten geworden
sind und mir infolgedessen nicht zugänglich waren. Das fällt jedoch
nicht so sehr ins Gewicht, da wir aus dem wenigen, das wir feststellen
können, sehen, ein wie geringe Kenntnis der Ritterromane der Verfasser
besass.

Fast alles, was wir vom {\it Amadis de Gaula}\pfn{2}{\pxxxviiinii} in dem falschen
{\it Don Quijote} finden, ist durch Vermittlung des ersten Teiles hineingelangt.
Für ein Citat aus diesem Roman (Cap.~XXXII), dessen Held von einem
Zauberer durch einen Trank aus Sand und kaltem Wasser beinahe getötet
worden sei, ist nichts Entsprechendes im {\it Amadis de Gaula} vorhanden.
Vielleicht haben wir es mit einer parodistischen Nachahmung
einer ähnlichen Verzauberung im {\it Amadis} (I.~Cap.~19) zu tun. Zu der
Gefangenschaft der Urganda bei dem Zauberer Friston (Cap.~XXII) fehlt
ebenfalls ein direktes Vorbild. Im {\it Esplandián} (Cap.~121) befindet sich
Urganda in der Gewalt der Infantin Melia. Im {\it Amadis de Grecia} tritt
ein unbekannter Ritter auf, der sich als Infantin Gradafilea entpuppt.
Das könnte für die Burlerinaepisode als Vorbild gedient haben. Wenn
am Schluss des falschen {\it Don Quijote} berichtet wird, dass der Junker
\originalpage{39}
in Begleitung eines als Mann verkleideten Weibes weiter geabenteuert
habe, so finden wir etwas Ähnliches im {\it Florisel de Niquea} (III, Cap.~78),
wo Finistea dem Amadis de~Grecia in Männertracht dient. Ausführlich
werden citiert {\it Belianis de~Grecia:} das Abenteuer des Helden und eines
anderen Ritters (Av.\ Cap.~XII) mit einigen Wilden und die Raserei
Rolands (Cap.~VI) nach dem {\it Espejo de Caballerías.} Das in Cap.~XXVI
berichtete Abenteuer eines griechischen Prinzen in einem verzauberten
Schloss\pfn{1}{\pxxxixni} erinnert an ein von Cirongilio de~Thracia bestandenes
Abenteuer (cf.~{\it D.~Q.}\ I,~32). Sonst erwähnt Avellaneda nur Namen, aber auch
da ist er nicht genau, z.~B.~{\itquoted Amadis de~Gaula}, {\itquoted don Belianis de~Grecia
y su hijo~(!)\ Esplandián} etc.\ (Cap.~II). (Esplandián ist der Sohn des
Amadis). Die mit ihren eisernen Keulen den Schlosseingang versperrenden
Riesen (Cap.~IX) sind mir nur aus dem {\it Huon de~Bordeaux}\pfn{2}{\pxxxixnii} bekannt,
der meines Wissens nicht in das Spanische übersetzt worden ist.

Als Ersatz für seine geringe Kenntnis der Ritterbücher hat Avellaneda
umso reichlicher die Romanzen benutzt. Meiner Meinung nach
eignen sich diese in nationalem Geiste gehaltenen Dichtungen wenig zu
einer Satire, die es hauptsächlich auf das höfische Liebes- und Lebensideal
der Ritterromantik abgesehen hat. Allerdings sind auch höfische
Elemente in die spätere spanische Romanzenpoesie eingedrungen.

Personennamen aus den {\it Cidromanzen} kommen in dem Pseudo-{\it Don
Quijote} häufig vor (Cap.~II, III, VI etc.), besonders citiert ist die Eroberung
von~Zamora (Cap.~VI) und die Beleidigung des toten Cid durch einen Juden,
und dessen Bestrafung (Cap.~VI). Ferner werden angeführt die Romanzen,
die die Eroberung Spaniens durch die Mauren behandeln (Graf~Julian,
Don~Rodrigo, Florinda, Pelayo, Sandoval etc.) und die dem Karolingercyklus
angehörigen Romanzen von Calaínos ({\it\spanish Ya cabalga Cataínos}\pfn{3}{\pxxxixniii}
etc.\ Cap.~VII vgl.\ {\it D.~Q.} I,~9). Aus dem Gedichte von dem {\it Marqués de
Mantua} und {\it Baldovinos} stammt der aus {\it D. Q.}~I,~10 bekannte Schwur\pfn{4}{\pxxxixniv}
des Grafen (Av.~Cap.~IV und~XII). Der Sekretär parodiert ihn (Cap.~XII)
in folgender Weise: {\itquoted\spanish Yo juro por el orden de secretario que recebí, de
no comer pan en el suelo ni folgar con la reina de espadas, copas,
\originalpage{40}
bastos ni oros, ni dormir sobre la punta de mi espada, hasta tomar tan
sanguinolenta venganza del principe don~Quijote de~la Mancha, que
los brazos le queden colgados de los hombros, y las piernas y muslos
asidos á las caderas, y la cabeza se le ande á todas partes, y la boca,
á pesar de cuantos ni han nacido ni han de nacer, le ha de quedar
debajo de las narices.} Endlich wird noch die Romanze über den Fall
Trojas erwähnt (Cap.~VIII): {\itquoted\spanish Fuego suena, fuego suena, que se nos alza
Troya con Elena.}

Eine ganze Menge von Namen und Citaten hat die Heiligenlegende
und die biblische Geschichte geliefert und zwar in so reichlicher Menge,
dass man Avellanedas {\it Don Quijote} dem echten als den orthodoxen
gegenüber gestellt hat. Das Altertum ist fast nur durch Namen vertreten.
Von der zeitgenössischen Litteratur nennt Avellaneda Lope de
Vega {\it (Testimonio vengado, Filis, Celia, Lucinda).} Cervantes (Escarraman,
Cap.~XXXI, aus des Dichters Entremés {\it El rufián viudo\/}), von der älteren
die {\it Celestina}, Ariost und Petrarca. ---

\blankline

Wie Cervantes in seinem ersten Teil die Novelle {\it\spanish El curioso impertinente}
eingefügt hat, hat auch Avellaneda es sich nicht nehmen lassen,
seinen {\it Don Quijote} mit {\emph Novellen} auszuschmücken. Die erste trägt
den Titel {\it\spanish El rico desesperado.}

Ein junger Student der Rechte zu Lovaina (Löwen), der ein wüstes
Leben führt, wird durch die Predigt eines Dominikaners bekehrt und
beschliesst, in ein Kloster einzutreten. Wie das Noviziat seinem Ende
zugeht, erhält er den Besuch zweier Freunde, die ihn für die Welt
wiederzugewinnen suchen. Trotz der Bemühungen des Priors, ihn zu
halten, kehrt er zu dem Weltleben zurück und vermählt sich einige
Zeit später mit einer vermögenden und tugendhaften Dame, die sich
ebenfalls bis dahin in einem Kloster aufgehalten hat. Nach dreijähriger
glücklicher Ehe bietet sich ihm Gelegenheit, eine Gouverneursstelle zu
erhalten, die durch den Tod seines Onkels frei geworden ist. Er reist
deshalb nach Brüssel gerade zu einer Zeit, wo seine Frau erwartet,
Mutter zu werden. Auf der Rückreise lernt er einen spanischen Soldaten
kennen und bietet ihm Gastfreundschaft in seinem Hause an. Dort
orfährt er, dass seine Frau eben einem Knaben das Leben gegeben hat.
Man speist deshalb am Bett der Wöchnerin. Der Anblick der schönen
Frau erregt in dem Soldaten verbrecherische Begierden. Der Umstand,
dass er im Nachbargemach schläft, der Gatte aber ein ferneres Zimmer
inne hat, lässt einen teuflischen Plan in ihm reifen. Er begibt sich bei
\originalpage{41}
Nacht zu der Wöchnerin, die ihren Gatten zu empfangen glaubt, dem
sie wegen seiner Rücksichtslosigkeit Vorwürfe macht. Nachdem der
Soldat sein Ziel erreicht hat, ohne erkannt zu werden, bricht er am frühen
Morgen auf. Wie die Gattin ihrem Mann gegenüber scherzhaft auf den
nächtlichen Besuch anspielt, versteht er sie zunächst nicht, dann ahnt
er den wahren Zusammenhang. Ohne sich etwas anmerken zu lassen,
eilt er dem Soldaten zu Pferd nach. Er holt ihn ein und ersticht ihn.
Inzwischen hat auch die Frau durch ein Selbstgespräch ihres Mannes,
das ein Stallknecht angehört und wiedererzählt hat, den wahren Sachverhalt
erkannt. In ihrer Verzweiflung stürzt sie sich in einen Brunnen
und sühnt so den unbewussten Ehebruch. Wie der zurückkehrende
Gatte ihren Tod erfährt, zerschellt er seinen neugeborenen Sohn am
Brunnenrand und gibt sich auf dieselbe Weise wie sein Weib den Tod.

Wenn wir von der Brutalität, die sich in dieser Erzählung ausspricht,
absehen, so ist sie das Beste, was Avellaneda geschrieben hat.
Der Stil ist sorgfältiger als sonst. Namentlich kommt die knappe Darstellung
der Katastrophe der tragischen Wirkung zu gute. Die Art, wie
der etwas anstössige Stoff behandelt wird, ist echt spanisch. Schon der
fatalistische Grundgedanke, dass der unglückliche Ausgang eine Strafe
für das Verlassen des Klosters ist, entspricht vollkommen den religiösen.
Anschauungen, die damals in Spanien herrschten. Auch die Selbstrache
erscheint hier vollkommen berechtigt. Ein jeder Spanier, der in einem
Glied seiner Familie beleidigt ist, ist selbst der {\quoted Arzt seiner Ehre}. So
musste dieser Stoff, ein unbewusster Ehebruch, der den Gatten zur Rache,
das Weib zur Sühne verpflichtet, mit seinem tief tragischen Gehalt besonders
den Spaniern zusagen.

Das gleiche Motiv ist auch sonst noch in der Weltlitteratur verbreitet.
Die berühmteste Fassung ist die Erzählung Boccaccios\pfn{1}{\pxlini} von
Agilulf und dem Stallknecht, die den Stoff in humoristischer Weise
behandelt: {\itquoted\italian Un pallafrenier giace con la moglie d'Agilulf re, di che
Agilulf tacitamente s'accorge: truovolo e tondelo: il tonduto tutti gli
altri tonde, e cosí campa dalla mala ventura} (Giornata Terza, Novella
Seconda). Von den bei Landau\pfn{2}{\pxlinii} verzeichneten Bearbeitungen desselben
Stoffes stehen unserer Novelle am nächsten: die altenglische Ballade
{\it Glasgerion} und die schottische {\it Glenkindie,} wo die betrogene Frau sich
entleibt und der Liebhaber den schurkischen Diener und sich selbst
\originalpage{42}
tötet.\pfn{1}{\pxliini} Noch mehr Verwandtes enthält die dreiundzwanzigste Erzählung
des {\it Heptameron} (bei Landau nicht genannt): {\it\french Trois meurtres advenuz en
une maison: à sçavoir en la personne du seigneur, de sa femme, et de
leur enfant par la méchanceté d'un Cordelier.} Entweder hat diese
Novelle Avellaneda als Quelle gedient oder beide Verfasser haben die
gleiche Quelle benutzt. Nicht erwähnt sind von Landau eine Anzahl
spanischer Komödien, wo das Motiv des unfreiwilligen Ehebruchs wiederkehrt.
Den Ausgangspunkt der Handlung bildet es in {\it\spanish El bastardo de
Ceuta} von Juan Grajales: Elvira, die Gattin des Hauptmanns Melendez,
ist einst in der Dunkelheit, als sie ihren Gatten zu empfangen glaubte,
von dem Fähnrich Gomez de~Melo umarmt worden. Ein Sohn ist die
Frucht dieses Ehebruchs. Der Gatte rächt sich erst viele Jahre später.
Der gleiche nächtliche Betrug spielt eine Rolle in {\it\spanish El Medico de su amor}
von Francisco de~Rojas Zorilla, in {\it\spanish La Prudencia en el Castigo} von
Lope de~Vega und den {\it\spanish Audiencias del rey Don Pedro,} die Schack
Lope de~Vega zuschreibt. In der letzten Komödie rächt sich die
beleidigte Frau selbst. Zu komischen Effekten ist das Motiv verwendet
worden in Ricardo de~Turias {\it\spanish La Burladora burlada.}

Am berühmtesten ist jedenfalls der nächtliche Betrug, den Don
Juan Tenorio im {\it\spanish Burlador de Sevilla} von Gabriel Tellez verübt, einmal
an Isabella, der Verlobten des Herzogs Octavio, das zweite Mal an Doña
Aña, Don Gonzalos Tochter und Braut des Marqués de la Mota. Da dieses
Stück hinreichend bekannt ist, kann eine nähere Besprechung unterbleiben.

Ebenfalls in katholisch-religiösem Geiste gehalten ist die andere
Erzählung: {\it\spanish Los Felices Amantes} (Cap. XVII--XX).

Don Gregorio sieht in einem Nonnenkloster die schöne Doña Luisa,
die trotz ihrer 25~Jahre bereits Priorin ist. Er kennt sie schon von
seiner Jugend her. Jetzt erwacht in ihm Liebe zu ihr. Ein Dienst, den
er ihr erweist, verschafft ihm eine längere Unterredung mit ihr, im
Laufe deren er ihr seine Liebe gesteht. Die Sprache, die er führt, ist
der galante Konversationston, geschraubt und geziert bis zur
Unverständlichkeit\pfn{2}{\pxliinii}. Wir sehen, dass die Nonnen damals sehr gern die
\originalpage{43}
Liebeserklärungen junger Kavaliere anhörten\pfn{1}{\pxliiini}. In einer zweiten
Unterredung am nächsten Tag bekennt die Priorin, dass sie seine Liebe
erwidere. Nach sechs Monaten ist die gegenseitige Neigung derart
angewachsen, dass Luisa sich erbietet, mit Gregorio zu fliehen. Um
Subsistenzmittel zu schaffen, greift sie den Klosterschatz an, er das
Eigentum seiner Eltern. Ehe sie das Kloster verlässt, kniet sie noch
einmal vor dem Bilde der Jungfrau, der sie sehr ergeben ist, nieder und
übergibt ihr die Schlüssel des Klosters. Dann flieht sie mit ihrem
Geliebten nach Lissabon, wo sie ein so üppiges Leben führen, dass
ihr Vermögen bis auf einen kleinen Rest, den Don Gregorio noch verspielt,
vollkommen aufgezehrt ist. Zu Fuss wandern sie nach Badajoz
in Kastilien, wo sie im Hospital aufgenommen werden. Luisa will sich
ihr Leben als Wäscherin verdienen. Der Verwalter des Hospitals wirbt
um ihre Liebe. Don Gregorio veranlasst seine angebliche Frau, ihn zu
erhören, da er Vorteil daraus zu ziehen hofft. Luisa sinkt tiefer und
tiefer. Sie wird öffentliche Dirne. Dies Leben dauert so lange, bis der
Sohn eines angesehenen Bürgers vor ihrem Haus erstochen wird. Luisa
wird deshalb in Haft genommen, aber durch die Vermittlung des
Hospitalverwalters bald wieder frei gelassen, Don Gregorio muss die
Stadt verlassen. Er begibt sich nach Madrid, wo er in den Dienst
eines Edelmannes tritt. Inzwischen hat sich Luisa, die des Kurtisanenlebens
überdrüssig geworden ist, entschlossen, in ihr Kloster zurückzukehren
und dort ihre Schuld zu büssen. Nachts verlässt sie, als
Pilgerin gekleidet, Badajoz und wandert nach ihrem alten Kloster. Um
Mitternacht kommt sie, nachdem sie unterwegs viel Entbehrungen
gelitten, an die Klosterkirche. Sie findet die Tür offen und ihren
Schlüsselbund am Altar, wo sie ihn hingelegt hatte. Da hört sie, wie
das Marienbild sie beim Namen ruft. Die Jungfrau verkündet ihr, dass
sie während ihrer vierjährigen Abwesenheit ihre Stelle als Priorin eingenommen
habe. Sie soll sich auf ihre Zelle begeben und ihr klösterliches
Gewand anlegen. Am nächsten Morgen sieht Luisa, dass niemand
sie vermisst hat. Sie beichtet dem Beichtvater des Klosters ihre Schuld
und das durch Maria bewirkte Wunder.

Auch an Gregorio ist die göttliche Gnade offenbar geworden.
Durch die Predigt eines Dominikaners, die das Schicksal des Mannes
(offenbar des Theophilus) behandelt, der sich dem Teufel verschrieben
\originalpage{44}
hatte, durch Maria aber vor der Verdammnis gerettet wurde, wird sein
Gewissen erweckt. Er beichtet und zieht nach Rom, um dort Vergebung
für seine Sünden zu erlangen. Von dort zurückgekehrt, begibt er sich
nach dem Kloster Luisas. Auf seine Frage nach der entlaufenen Nonne
erhält er die Antwort, dass sie noch Priorin und hoch geachtet wegen
ihres heiligen Lebens sei. Ganz verwirrt durch diese Auskunft, geht
er nach dem Haus seiner Eltern, denen er --- ein zweiter Sankt Alexius ---,
ohne sich zu erkennen zu geben, Nachricht von ihrem Sohn bringt, den
er angeblich in Neapel gesehen hat. Da erfährt er auch, dass seine
Eltern das Geld, das er ihnen entwendet hatte, nicht vermisst haben.
Maria hat es ebenso wie das aus dem Klosterschatz entnommene Geld
durch ein Wunder ersetzt.

Schliesslich wird er von seiner Mutter erkannt. Er bleibt aber in
seinem Entschluss, ein Mönch zu werden, fest. Von der Priorin hört
er, durch welches Wunder ihre Abwesenheit geheim geblieben ist. Er
tritt in ein Kloster ein, wird Prälat desselben und stirbt am selben
Tage und zur selben Stunde wie die Priorin.

Diese Erzählung ist eine novellistische Bearbeitung des als
{\emph Beatrixlegende} bekannten Marienwunders\pfn{1}{\pxlivni}.
Diese Legende ist im Mittelalter seit Caesarius von~Heisterbach~(1223)\pfn{2}{\pxlivnii}
öfters lateinisch\pfn{3}{\pxlivniii} und vulgärsprachlich\pfn{4}{\pxlivniv}
und in neuerer Zeit von Charles Nodier\pfn{5}{\pxlivnv}, Gottfried Keller
(in den sieben {\it Legenden\/}) und Maurice Maeterlinck (als Drama)\pfn{6}{\pxlivnvi}
behandelt worden. Sie gehört zu den Marienwundern, die die Ansicht
vertreten, dass die Mutter Gottes auch dem grössten Sünder für den
geringsten ihr erwiesenen Dienst ihre Hilfe und ihre Gnade gewährt.
Watenphul (a.~a.~O.) hat das Verhältnis der mittelalterlichen Fassungen
studiert. Er sucht sie alle auf die von Caesarius Heisterbacencis
gelieferte Form zurückzuführen und unterscheidet in der weiteren
Entwicklung der Legende zwei Gruppen:
\originalpage{45}

1. Die direkte Descendenz von Caesarius von~Heisterbach.

2. Die Form, die aus der Verschmelzung der Originallegende mit
dem Typus {\quoted Nonne aus dem Kloster} (Mussafia) entstanden ist: {\quoted Die
Nonne wird beim Verlassen des Klosters zweimal von der Jungfrau Maria
aufgehalten, bis es ihr das dritte Mal, wo sie ohne Gruss an dem Muttergottesbilde
vorbeischreitet gelingt, das Kloster zu verlassen.} Zu dieser
Gruppe können wir noch eine Legende stellen, die auch Legrand-d'Aussy\pfn{1}{\pxlvni}
zur Beatrixlegende rechnet: {\quoted Eine Nonne (weder als Küsterin, noch mit
Namen bezeichnet) wird von dem Neffen der Äbtissin verführt. Im
übrigen Übereinstimmung mit der Legende von Beatrix der Küsterin.}

Die Quellenbestimmung für die Novelle Avellanedas ist durch den
Hinweis auf den {\it Discipulus}\pfn{2}{\pxlvnii} nicht ohne weiteres erledigt. Denn seine
Erzählung bringt weit mehr als der {\it Discipulus}\pfn{3}{\pxlvniii}, der nur Caesarius
kopiert. Der Inhalt dieser Fassung ist folgender:

{\quoted Die Küsterin Beatrix, eine glühende Verehrerin der Jungfrau Maria,
verlässt das Kloster, um ihrem Buhlen zu folgen, und vertraut die Schlüssel
der Maria an. Von ihrem Verführer verlassen, lebt sie 15~Jahre als
Dirne. Endlich kehrt sie zurück; niemand hat ihre Abwesenheit bemerkt,
denn Maria hat ihre Stelle vertreten.}

Das Motiv der Entwendung des Geldes durch die Liebenden und
die wunderbare Zurückerstattung durch die Jungfrau ist der Legende
{\it Rittersfrau und Kleriker}\pfn{4}{\pxlvniv} entlehnt. Die für die ganze Descendenz von
Caesarius von~Heisterbach charakteristische Frage der zurückkehrenden
Nonne an der Klostertür bezw.\ im Nachbarhause fehlt, dafür steht die
Erkundigung Gregorios. Die Legende beschäftigt sich nicht mit den
weiteren Schicksalen des Verführers. Der ganze die Rückkehr Gregorios
behandelnde Teil bei Avellaneda ist unter deutlichem Einfluss der
{\it Alexiuslegende} entstanden. Gemeinsam mit einigen jüngeren Fassungen
\originalpage{46}
(altisländische\pfn{1}{\pxlvini}, mittelniederländische\pfn{2}{\pxlvinii} und Nodier),
jedenfalls aber selbstständig von Avellaneda eingeführt, ist die Annahme einer
Jugendbekanntschaft zwischen der Priorin und dem Jüngling. Die Personennamen und
die Lokalisierung der Erzählung stammen auch von ihm. Einzelheiten,
die Gespräche der Liebenden, der Aufenthalt in Lissabon, in Badajoz usw.,
sind novellistisch ausgestaltet, so dass das alte Legendarische fast
vollständig überwuchert ist. Nach Abzug des Legendenhaften hat unsere
Novelle eine gewisse Ähnlichkeit mit François Prévosts {\it Manon Lescaut}
(1728). Der Held dieses Romans bleibt, nachdem er den Rest seines
Vermögens verspielt hat, bei seiner Geliebten, die sich und ihn durch
ihre Schande erhält.

Um das Verzeichnis der von Watenphul angegebenen Fassungen
zu vervollständigen, will ich noch zwei spanische Bearbeitungen anführen:

1. Calderon de la Barca im {\it Purgatorio de San Patricio.} Ludovico
Enio erzählt dem König Egerio in Jornada I.\ dieses Stückes, er habe
eine Nonne verführt, geraubt und geheiratet, sich mit ihr nach Valencia
begeben, dort, nachdem er sein Vermögen verschwendet habe, den Versuch
gemacht durch ihre Unehre Geld zu gewinnen. Sie jedoch habe sich
geweigert und sei in das Kloster zurückgeflohen. --- Es fehlt das Wunder.

2. Lope de~Vega hat die Legende zum Gegenstand eines Schauspiels
{\it\spanish La Buena Guarda} gemacht.

Doña Clara de Lara, Oberin eines Klosters, steht im Geruche der
Heiligkeit, lässt sich aber von ihrem Majordomus Felis zur Liebe und
Flucht verführen. Ihre Stelle im Kloster nimmt während ihrer Abwesenheit
ein Engel in ihrer Gestalt ein, welcher ihr, nachdem sie bereut und
Busse getan hat, die Regierung wieder übergibt.

Die diesem Stück zu Grunde liegende Fassung der Legende wird
repräsentiert durch die Handschrift Brit.\ Mus.\ Additional 33\,956\pfn{3}{\pxlviniii}
(Anf.\ 14.~Jahrh.), wo ebenfalls ein Engel die Vertretung übernimmt\pfn{4}{\pxlviniv}.
\originalpage{47}
Ein abschliessendes Urteil über Avellanedas {\it Don Quijote} lässt sich
in wenigen Worten geben. Dass er dem Werke des Cervantes bei
weitem nachsteht, wird nach den vorausgegangenen Betrachtungen keinem
Zweifel unterliegen. Trotzdem muss jeder zugeben, dass man sich bei
seiner Lektüre ganz gut unterhält. Die Komik ist reichlich aber derb.
Tubino\pfn{1}{\pxlviini} fasst seine Meinung in folgenden Worten zusammen, denen wir
gern beistimmen: {\itquoted\spanish Es el {\emph Quijote} de Avellaneda una novela
entretenida: el {\emph Quijote} de Cervantes, simulacro eterno de la humanidad
en todas las zonas, en todos los tiempos y en todas las gradaciones y
esferas de la vida. Distrae el primero haciendo reir, el segundo lleva
la melancolía al animo y pone lágrimas en los ojos.}

Günstiger lautet das Urteil Fitzmaurice-Kellys: {\itquoted\english It is, in fact, a work
of considerable interest and entertainment and, were Cervantes not in
possession of the field, it would still find readers.} Und wäre der
{\it Don Quijote} des Cervantes nicht geschrieben worden, können wir sagen, so
hätte Avellaneda nicht den seinen verfasst.

\centerline{\emph{Zweiter Teil.}}

{\bf Lesages Bearbeitung von Avellanedas {\it Don Quijote.}}

Nachdem Lesage sich mit Übersetzungen aus dem Spanischen
({\it Théâtre espagnol,} 1700) auf dem Gebiet des Dramas versucht hatte,
führte er sich auch als Romanschriftsteller mit einer Übersetzung aus
dem Spanischen ein. So bezeichnet er wenigstens seine Bearbeitung
des {\it Don Quijote} von Avellaneda, obgleich der erste Band eine sehr
freie Übertragung, der zweite aber fast vollständig originell ist.

Einen besonderen Erfolg hatte er nicht mit seinem Buche, wenn
es auch das {\it Journal des Sçavants} (1704 S.~207) sehr günstig beurteilte:
{\itquoted\french Les lecteurs français ne s'apercevront point de la rudesse d'Avellaneda
dans la traduction qu'on donne aujourd'hui au public; le style en
est aisé et sans embarras. On est obligé à l'auteur du soin qu'il a
pris de donner à sa traduction un tour et des manières de parler si
françaises qu'on n'y reconnaît plus les défauts que Cervantes trouvait
dans l'original. Mais il est à craindre qu'on ne dise qu'il est tombé
dans un autre défaut; c'est de répéter trop souvent certaines manières
de parler populaires, comme Par la gerny, Mardy, oh! dame! et
\originalpage{48}
plusieurs autres. Il est vrai que c'est dans la bouche de Sancho qu'il
les met; mais ne pourrait-on point dire qu'elles y sont trop fréquentes
et qu'on en est fatigué?}

1. Den sechsunddreissig Kapiteln Avellanedas entsprechen siebzig
Kapitel bei Lesage. Mit dem Fortschreiten der Erzählung entfernt sich
der Bearbeiter immer mehr von seiner Vorlage.

Avellanedas Capitulo I --- entspricht den Kap.~1 und~2 und einem
Teil von~3. Sancho bringt am Sonntag nach seinem ersten Besuch den
Ritterroman Floristian de Candaria, durch dessen Lektüre in dem Ritter
die alte Narrheit erwacht. Er beschliesst den vierten Auszug, verspricht
Sancho einen neuen Esel und schickt ihn mit einem Brief zu Dulcinea.
Ausgelassen ist nur Unwesentliches.

Capitulo II. --- Rest von Kap.~3 und~4 (zum Teil). Die Antwort
Dulcineas ist um eine Drohung mit ihren Brüdern erweitert. Ziemlich
wörtlich.

Capitulo III. --- Kap.~4 und~5. Hinzugefügt ist ein Citat aus dem
{\it Belianis de Grecia:} {\itquoted\french Ce fut de cette sorte et par le ministère de
l'Infante Impéria que la Sage Belonie fit tenir des armes á Don
Belianis son favori} etc. Sancho kauft den Esel des Tomé Cecial
({\it D.~Q.}~II). Die alte Lanze finden sie als Besenstiel wieder. Eine
kupferne Waschschüssel ist der {\quoted Schild des Bandenazar}. Erwähnung
Sanson Carrascos ({\it D.~Q.}~II).

Capitulo IV und V --- Kap.~6.

Capitulo VI --- Kap.~7 und~8. --- {\itquoted\french Le palais magnifique qui
s'offroit tantôt à ma vûe est disparu \dots il ne faut pas s'étonner, si
toy qui n'es qu'un païsan, tu ne vois les choses qu'en païsan. Mais
moy qui suis armé Chevalier, et qui par conséquent voy les choses comme
elles sont réellement; j'ay sujet d'estre surpris de n'appercevoir ici
qu'une simple cabane.} (cf.~L.~5.~Cap.~32.)

Capitulo VII --- Kap.~9--11. --- Bei Tafel spricht man über den
1.~Teil des Don Quijote von Cid Hamet Benengely. Don Quichotte
durchblättert ihn und bricht in Zorn aus über das Bild, das der Verfasser
von ihm macht (cf.~{\it D.~Q.}~II,~59). Dann spricht er über Poesie
und recitiert ein Sonett an Dulcinea.

Capitulo VIII --- Livre second. Kap.~12.

{\itquoted\french Infortuné Chevalier, s'écrie le sage Alisolan au commencement
de ce Chapitre} etc.\ (ahmt ähnliche Kapiteleingänge nach, {\it D.~Q.}~I,~15,
22, II~8, 10, 24, 27 etc.).
\originalpage{49}

Capitulo IX und X --- Kap.~13 und~14.

Capitulo XI --- Kap.~15. Turnier in Zaragoza stark gekürzt. Es
fehlen die Mottos der Ritter.

Capitulo XII und XIII --- Kap. 16--18.

Capitulo XIV --- Livre Troisième. Kap.~19.

Capitulo XV und XVI enthalten die Novelle {\it\spanish El rico desesperado,}
die bei Lesage fehlt.

\longdash Kap.~20. {\it\french De la mort du Frère Jacques et de ce qui se
passa à son enterrement.} Ein verstorbener Einsiedler, Frère Jacques,
wird als verkleidete Frau erkannt. Bei ihrem Anblick fällt Bruder
Stephanus, der Eremit Avellanedas, in Ohnmacht. Im Pfarrhause des
nächsten Dorfes erzählt er zur Begründung seiner Ergriffenheit die
Novelle {\it Los Felices Amantes} (Kap.~21 und~22). Er ist Don Gregorio,
der verstorbene Einsiedler Luisa. Das Legendenhafte ist ausgelassen.

Auf diese Weise wird die Novelle an die Hauptfabel angeschlossen.

Capitulo XXI u.\ XXII --- Kap. 23--25 (z.~T.).

Capitulo XXIII --- Kap.~25 (Forts.) ist von Lesage von Obscönitäten
gesäubert worden.

\longdash Kap.~26. Man trifft eine Kutsche. Ihre Insassen sind der
Bruder Antonios de Bracamonte (Cap.~XIV), der aus Peru zurückkehrt,
dessen Frau und seine Schwiegermutter (cf.\ das Wiedersehen des Richters
und des Kapitäns in {\it D.~Q.}~I).

\longdash Kap.~27. {\it Histoire de Don Raphaël de~Bracamonte.} Don
Raphael ist zu der Zeit nach Peru gekommen, wie sich Pizarro dort der
Herrschaft bemächtigt. Er schliesst sich nicht ihm, sondern dem
königstreuen Offizier und Gouverneur der Insel Caxamalca Melchior Verdugo
an, dessen Freund er wird. Während man erfolglos gegen Pizarro kämpft,
kommt aus Spanien der Licenciat Pierre de~la~Gasca als Präsident des
Kgl.~Gerichtshofes. Diesem unterwerfen sich einige von Pizarros Offizieren.
Pizarro selbst wird bei Xaguixguana besiegt. Zur Belohnung
für seine Dienste erhält Don Raphael einige Indianer und gründet eine
Silbermine, die ihm in acht Jahren 100\,000 Taler einbringt. Mit diesem
Gelde schifft er sich nach Lima ein. Bei Panama erleidet er Schiffbruch,
rettet nur das nackte Leben und wird von einem Don Michael aufgenommen.
Er gewinnt die Liebe einer begüterten Dame, Doña Theodora,
der Tochter Doña Marias. Ehe die Mutter ihm ihre Tochter gibt,
erzählt sie ihm ihre Geschichte.

Sie hat sich, um ihren Bruder, der den Neffen eines Gouverneurs
getötet hatte, zu retten, dem Gouverneur preisgegeben. Daher stammt
\originalpage{50}
