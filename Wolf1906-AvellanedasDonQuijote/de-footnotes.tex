\def\piini{
Nach Perez Pastor; vgl.\ dazu Morel-Fatio, {\it Bull. Hisp.}~V,~4, S.~361.
}

\def\piinii{
Das Wichtigste ist veröffentlicht in der Biographie von Cayetano Alberto
de~la~Barrera ({\it Obras} de Lope de~Vega publicadas por la Real Academia Española,
Madrid~1890~ff., I.~Bd.).
}

\def\piiniii{
Zweite Auflage von Francisco Perez~Bayer, Madrid 1783--88: {\itquoted\latin Alphonsus
Fernandez de Avellaneda, patria ex oppido Tordesillas Pincianae diœcesis,
continuavit sed absque genio illo qui principem Michaelis Cervantes ad inventionem
promovit et comitatus est.}
}

\def\piiini{
cf.~Vorrede zu seiner Übersetzung; die dort erwähnte Legende,
dass die Freunde des Cervantes alle erreichbaren Exemplare des falschen {\it Don Quijote}
verbrannt hätten, scheint mir auf {\it D.~Q.}\ II,~70 zu beruhen. (Altisidoras Traum:
die Teufelchen spielen mit Avellanedas {\it Don Quijote} Ball und werfen ihn dann
in den Abgrund der Hölle.)
}

\def\piiinii{
{\it\french Histoire comparée des littératures française et espagnole,} 1843, Bd.~I, S.~29.
}

\def\piiiniii{
1. {\spanish {\emph Segundo} \vvbar {\emph Tomo Del} \vvbar
{\emph Ingenioso Hidalgo} \vvbar {\emph Don Quixote
de la Mancha,} \vvbar que contiene su tercera salida: y es la \vvbar
quinta parte de sus aventuras.\ \vvbar Compuesto por el Licenciado Alonso Fernandez
\vvbar de~Avellaneda,
natural de la Villa de Tordesillas.\ \vvbar Al Alcalde, Regidores, y hidalgos, de la
noble villa del~(!) Argamesilla, patria feliz del hidalgo Cauallero Don Quixote \vvbar
de la Mancha.\ \vvbar (Grabadito: Caballero embistiendo lanza en
ristre\subpfn{1}{--- {\it D.~Q.} Valencia,~1605.}) \vvbar Con
Licencia. En Tarragona, en casa de Felipe \vvbar Roberto,
Año~1614.}\subpfn{2}{War mir nicht zugänglich. cf.~Leopoldo Rius, a.~a.~O.~II, S.~255.}

Die Aprobacion des Doctor Raphael Orthoneda ist datiert auf den 18.~April
1614, die Druck- und Verkaufserlaubnis für das Erzbistum Tarragona auf den
4.~Juli 1614.

Es folgt eine Dedikation: al Alcade, Regidores etc.

Vorrede Avellanedas.

Sonett von Pero Fernandez.

Avellaneda teilt nach dem Muster des ersten Teiles ein und zählt weiter:
{\it quinta, sexta} und {\it septima parte,} während Cervantes seinen zweiten Teil
schlechtweg {\it Segunda Parte} nennt und weiter keine Unterabteilungen macht.

2. {\spanish {\emph Vida y hechos} \vvbar {\emph del Ingenioso Hidalgo} \vvbar {\emph Don Quixote} \vvbar
{\emph de la Mancha}, \vvbar que contiene su quarta~(!) salida, \vvbar
y es la quinta parte de sus
aventuras. \vvbar Compuesto por el Licenciado Alonso Fernandez \vvbar de Avellaneda,
natural de la villa de Tordesillas.\ \vvbar Parte~II.\ Tomo~III.\ \vvbar
nuevamente añadido, y corregido en esta \vvbar Impresion, por el Licenciado
Don Isidro Perales y Torres\subpfn{3}{Nach
D.~Juan de~Iriarte {\it (Adiciones á la Bibl. esp. de Nic. Antonio)} ---
Blas Antonio Nasarre y Ferriz (1689--1751).}. \vvbar
Dedicada al Alcalde etc.\ \vvbar Año 1732} (Der Holzschnitt, der das Abenteuer mit
dem Ritter vom {\quoted hellen Mond} darstellt, findet sich schon in der Ausgabe des
{\it D.~Q.}\ Madrid,~1730)\ \vvbar {\spanish Con Privilegio \vvbar En Madrid etc.}

Der Titel {\it Vida y hechos} ist seit der Brüsseler Ausgabe des {\it D.~Q.}\ (1662
und~1671) gebräuchlich. Die Avellanedaausgabe sollte als Fortsetzung für die
Madrider {\it Don Quijote}-Ausgabe von~1730 gelten. Die Aprobacion von Don Agustin
de Montiano y Luyando ist sehr günstig für Avellaneda.

3. Dieselbe Ausgabe, Madrid~1806, unter Auslassung der beiden Novellen
(Cap.~XV--XX). Der Herausgeber wird in seiner Vorrede dem Verdienst des
Cervantes etwas mehr gerecht als der Herausgeber von~1732.

4. Abdruck der Editio princeps im 18.~Band der {\it\spanish Biblioteca de Autores
Españoles} (Rivadeneyra) unter dem Nebentitel {\it\spanish Novelistas posteriores á Cervantes}
mit einer Vorrede von Don Cayetano Rosell. Madrid~1851 und~1898.

Letztere Ausgabe wurde von mir benutzt.

5. Barcelona,~1884, in der {\it\spanish Biblioteca clásica española,} Daniel Cortezo y~\Ca.
Merkwürdig berührt das Urteil: {\itquoted\spanish el Quijote de Avellaneda ha sido reputado
por algunos como libro clásico, y figura en la lista de autoridades del idioma.}
Ausserdem ist diese Ausgabe von anstössigen Stellen gesäubert.

6. {\spanish {\emph El Quijote apocrifo}, compuesto por el Licenciado Alonso Fernandez
de Avellaneda, natural de Tordesillas. Edición cuidadosamente cotejada con la
original, publicada en Tarragona en~1614. Barcelona, Impr.~de J.~Jepús. 1905.}
}

\def\pivni{
1. {\french {\emph Nouvelles avantures de l'admirable Don Quichotte de
la Manche,} composées Par le Licencié Alonso Fernandez de Avellaneda: Et
traduites de l'Espagnol en François pour la première fois. Paris, Chez la Veuve
de Claude Barbin, 1704. Avec Privilège du Roy. 2~vols en \duodecimo.
ferner Amsterdam~1705 und London~1707 (Par M.~L.~S.), Bruxelles~1707 (nur bei Claretie
a.~a.~O.\ S.~430), Paris~1716, David~1741 (citiert bei Quérard, {\it La France littéraire}).}

Diese Bearbeitung, die von Lesage herrührt, war bereits~1702 vollendet
(Gutachten von Fontenelle 25.~Oct.~1702). Das Privileg ging von Gabriel Martin
auf die Witwe Claude Barbin über, sodass sich der Druck um zwei Jahre verzögerte.
Näheres später S.~47~ff.

2. {\french {\emph Le Don Quichotte de Fernandez Avellaneda,} traduit de
l'espagnol et annoté par A.~Germond de Lavigne. Paris, Didier,~1853.}

Die Einleitung dieser Übersetzung war im vorhergehenden Jahre bereits
gesondert erschienen:

{\french {\it Les deux Don Quichotte,} étude critique sur l'œuvre de Fernandez Avellaneda
faisant suite à la première partie du Don Quichotte de Cervantes. P.~1852.}
}

\def\pivnii{
1. {\english {\emph The History of the Life and Adventures of the famous
Don Quixote}\fourdots Now first translated from the original Spanish. With a
preface giving an Account of the Work. By Mr.~Baker. 2~vols. London~1745.}

Diese mir nicht zugängliche Ausgabe ist nach Yardley, Vorrede S.~VI, aus
dem Französischen übersetzt.

2. {\english {\emph A Continuation of Don Quixote} etc. Translated into English by
William Augustus Yardley,~Esq. 2~vols. London~1784.}

3. {\english {\emph The Life and Exploits of the ingenious Gentleman Don
Quixote}. Swaffham~1805.}

Nach der Ausgabe von Madrid~1732.
}

\def\pvni{
1. {\english {\emph A Continuation of the Comical History of the most
ingenious knight Don Quixote de la Mancha.} By the Licentiate Alonso
Fernandez de Avellaneda. Being a third volume; never before printed in English,
Translated by Captain John Stevens. London~1705.}

2. {\emph Nieuwe Avantuuren van Don Quichot,} door Avellaneda. Utrecht~1706.

do.~Amsterdam~1718.

3. {\emph Neue Abentheuer des Ritters Don Quichotte,} geschrieben von
Avellaneda, aus dem Frantzösischen in die teutsche Sprache übersetzt.
Copenhagen~1707.

{\emph Leben und Thaten des weisen Junkers Don Quixote von La
Mancha}. Aus der Urschrift des Cervantes, nebst der Fortsetzung des
Avellaneda. In 6~Bänden von F.~I.~Bertuch, Leipzig~1775 u.~dass.~1780--81,
Bd~5/6.
}

\def\pvnii{Pope, {\it Essay on Criticism} v.~267~ff.:

{\obeylines\parindent=2\the\parindent%
\glqq Once on a time La Mancha's knight, they say,
A certain bard encount'ring on the way,
Discoursed in terms as just with looks as sage
As e'er could Dennis of the Grecian Stage;
Concluding all were desperate sots and fools
Who durst depart from Aristotle's rules.\grqq
}

Vgl.~Lesage, Chap.~25. % TODO
}

\def\pvniii{
Tauchnitz ed.~S.~22.
}

\def\pvniv{
Becker, {\it die Aufnahme des Don Quixote in die engl.~Litteratur.} Diss. Berlin 1903.
}

\def\pvini{
{\itquoted\spanish El autor de este Don Quijote
no es Alonso Fernandez de Avellaneda,
natural de Tordesillas, porque constando de lo que Cervantes dice que el autor
es aragonés y no habiendo lugar en Aragon que se llame Tordesillas, se debe
conjeturar que quien fingió su patria, fingiría al nombre} \dots
}

\def\pvinii{
{\it Vida de Cervantes,} Londoner Ausg.~1738.
}

\def\pviniii{
{\it Vida de Cervantes} S.~30~ff., Ausgabe der Kgl.~Span.~Akademie. Madrid,~1780.
}

\def\pviniv{
{\it Vida de Cervantes,} 1797.
}

\def\pviini{
In dem Festzuge gingen: {\it\spanish {\quoted Don Quijote de la Mancha con un traje de
burlas, arrogante y picaro; puntualmente de la manera, que en su libro se pinta \dots
Esta figura y otra de Sancho Panza, su criado, que le acompañaba causaron
grande regocijo y entretenimiento, porque á mas de que su vestimenta era en
estremo graciosa, lo era tambien la invencion que llevaban} etc.}

({\it\spanish Retrato de las fiestas que á la beatificacion de la bien aventurada virgen
y madre santa Teresa de~Jesus etc.} Por Luis Diez de Aux. Zaragoza,~1615.)
}

\def\pviinii{
cf.~Tubino, {\it Cervantes y el Quijote.} Madrid~1872, S.~18~ff.
}

\def\pviiniii{
Näheres über die bei derartigen Wettbewerben herrschenden Gebräuche
s.~Tubino, a.~a.~O.\ S.~23.
}

\def\pviiniv{
Navarrete, {\it Vida de Cervantes,}~1819, hat zu dem von Pellicer Gesagten
nichts Wesentliches hinzugefügt.
}

\def\pviiini{
{\it\spanish El Conde-Duque de~Olivares y el rey Felipe~IV.}\ (Cadiz~1846) und {\it\spanish Miguel
de~Cervantes y dos inquisidores}~(1872).
}

\def\pviiinii{
Tubino, a.~a.~O.\ S.~1--82.
}

\def\pviiiniii{
Zeitschriftenaufsätze, abgedruckt in Asensio, {\it Cervantes y sus obras.}
Barcelona~1902. S.~462~ff.
}

\def\pviiiniv{
Ludwig Braunfels in seiner Übersetzung des {\it Don Quijote,}
Jubiläumsausgabe~1904, und W.~v.~Wurzbach, {\it Lope de~Vega.} Leipzig~1899. S.~55.
}

\def\pviiinv{
{\it Revista de Literatura.} Sevilla 1856--1858.
}

\def\pviiinvi{
Einleitung zu {\it Bibl.\ de aut.\ esp.}~XVIII. S.~VIII.
}

\def\pviiinvii{
Biblioteca Nacional M.~200.
}

\def\pixni{
Über den Sturz Aliagas siehe Quevedo, {\it\spanish Grandes Anales de quince días.} ---
Fernandez-Guerra, {\it Obras}~I, S.~203.
}

\def\pixnii{
Dunkle Anspielungen darauf in Quevedo, {\it\spanish Epistolario} ({\it Obras}
ed.~Fernandez-Guerra S.~575) vgl.\ auch Tubino, a.~a.~O.\ S.~273.
}

\def\pixniii{
Von den Dominikanern zu Ehren des heiligen Hyacinthus veranstaltet,
cf.~Fitzmaurice-Kelly, {\it Life of Cervantes} S.~197.
}

\def\pixniv{
Dies ist die früheste Ausgabe, neugedruckt in {\it\spanish Semanario erudito} de Don
Antonio Valladares de~Sotomayor. Madrid~1787. S.~264~ff.
}

\def\pxni{
Fernandez-Guerra, {\it Obras}~I, S.~198.
}

\def\pxnii{
S.~265 in der {\it Venganza}:
\dots{\itquoted\spanish burlandose del mundo hasta dar con su
pluma en el infierno, sin temor de sacarla chamuscada}.

Der {\it Discurso} ist später betitelt worden:
{\it\spanish El entremetido y la dueña y el soplón.}
}

\def\pxniii{
{\it El Imparcial} (15.~Febr.~1897).
}

\def\pxniv{
Groussac, {\it\french Une énigme littéraire: le Don Quichotte d'Avellaneda.} Paris~1903.
}

\def\pxnv{
L., der Schildknappe Bruneos de~Bonamar cf.~{\it Amadis de Gaula}~III,~3,~13,
IV,~20 etc.
}

\def\pxini{
Cayetano Rosell, {\it Bibl.\ de aut.\ esp.}~18, S.~X drückt sich sehr unbestimmt
aus: {\itquoted\spanish Léase este folleto, léase el Quijote de Avellaneda y se hallará
el mismo estilo, las mismas locuciones; en una palabra, la misma pluma.}
}

\def\pxinii{
Die bekannteste ist das {\it\spanish Tribunal de la justa Venganza del Licenciado
Arnoldo de Franco-Furt,} das nach Fernandez-Guerra ({\it Obras}~I, S.~XC) von dem
Pater Niseno, dem Doktor Juan Perez de~Montalban, Don Luis Pacheco de~Narvaez
u.~a.~verfasst sein soll.
}

\def\pxiniii{
{\itquoted\spanish Pregmática que este año de~1600 se ordenó por ciertas personas deseosas
del bien común y de que pase adelante la república, sin tropezar ni usar de bordoncillos
inútiles pues se puede andar sin ellos y por camino llano, en la conversacion
y en el escribir de cartas, con que algunos tienen la buena prosa corrompida y enfadado
el mundo.} --- Fernandez-Guerra, {\it Obras}~I, S.~429.
}

\def\pxiniv{
Er erwähnt den {\it Buscon}~(1626), das {\it\spanish Capitulo de los gatos}~(1627) und das
damals noch nicht gedruckte Schriftchen {\it\spanish Gracias y desgracias del ojo del c.}
}

\def\pxinv{
Bezieht sich auf das {\it\spanish Memoriál \dots y las indulgencias concedidas á los
devotos de monjas.} Er nennt Rabelais und Jean Marot als solche, die ebenfalls die
Geistlichkeit verhöhnt haben, meint aber jedenfalls Clément Marot, der sich durch
seinen kaustischen Witz bei den Mönchen verhasst machte.
}

\def\pxiini{
Groussac, {\it\french Une énigme littéraire.} Paris~1903. S.~170~ff. Vgl.\ dazu die
Rezension von Morel-Fatio, {\it Bulletin hispanique}~1903. No.~4. S.~360.
}

\def\pxiinii{
{\it\spanish Vida de Cervantes}~1797.
}

\def\pxiiniii{
Das Aragonesische ist ein mit katalanischen Elementen durchsetztes
Patois.
}

\def\pxiiini{
{\it\spanish Diccionario de voces aragoneses} S.~75.
}

\def\pxiiinii{
Labernia y~Estellez, {\it\catalan Diccionari de la Llengua Catalana ab la
Correspondencia Castellana.} Vol.~I,~II, Barcelona, s.~a.\ verzeichnet
kat.\ {\it senjal} als m.~u.~f.
Der eine Fall bei Avellaneda ist vielleicht auch durch den Drucker verschuldet.
}

\def\pxivni{
Morel-Fatio verweist auf: P.~Juan Mir y~Noguera, {\it Frases de los autores
clásicos españoles.} Madrid~1899. S.~605.
}

\def\pxvni{
Auch von Cervantes im {\it Licenciado Vidriera} citiert.
}

\def\pxvini{
{\it Vida de Cervantes.} Cadiz~1872.
}

\def\pxvinii{
2.~März 1612: {\itquoted\spanish y lei unos versos con unos antojos de Zervantes que
parecían unos huevos estrellados mal hechos.}
}

\def\pxviniii{
{\itquoted\spanish Lope de~Vega, como Avellaneda, escriben frecuentemente sin artículos \dots
en usar palabras ó demasiado afectadas ó demasiado vulgares y bajas se igualan
asimismo.}
}

\def\pxviiini{
Höchstens könnte eine Erwähnung des {\it\spanish Castillo de San Cervantes} angeführt
werden, die einer Bemerkung in der Vorrede: {\itquoted\spanish es ya viejo como el castillo
de Cervantes} entspricht. cf.~Gongora, {\it Romanzen} ({\it Aut.\ esp.}~32, S.~513):

{\obeylines\parindent=2\the\parindent%
\glqq Castillo de San Cervantes
Tú que estás junto á Toledo\grqq etc.
}
}

\def\pxviiinii{
{\itquoted\spanish Pintaos comedor y simple, y no nada gracioso, y muy otro del Sancho
que en la primera parte de la historia de vuestro amo se describe.} (II,~59.)
}

\def\pxxni{
Anders Fitzmaurice-Kelly, {\it Life of Cervantes,} S.~267: {\itquoted\english The resemblances
between the two may be accounted for by Cervantes' unwary habit of reading to
others what he had written long before it was ready for the press.}
}

\def\pxxini{
D.~Q.\ hält sie für ein Schloss, cf.~I,~2.
}

\def\pxxinii{
I,~16.
}

\def\pxxiniii{
I,~29.
}

\def\pxxiniv{
I,~3 und~17.
}

\def\pxxinv{
cf.~I,~25; in I,~31 berichtet Sancho, er habe Dulcinea angetroffen, wie sie
Getreide worfelte: bei Avellaneda ist sie dabei Mist aufzuladen, wovon der gute
Schildknappe als Dank für die Botschaft eine Portion ins Gesicht bekommt.
}

\def\pxxinvi{
Orlando ist nur an der Fusssohle verwundbar: I,~26 {\itquoted\spanish\dots no le {\rm (Orlando)}
podía matar nadie sino era metiendole una alfiler de á blanco por la planta del pié.}
}

\def\pxxinvii{
Darüber grosse Klage Sanchos cf.~I,~23.
}

\def\pxxiini{
Folgende Situation ist aus dem {\it D.~Q.}\ entlehnt:
TODO FIXME BUG HACK
}

\def\pxxiinii{
Die Vorbedingungen für dieses Abenteuer sind ganz ähnliche wie die für
Don Quijotes Kampf mit den Weinschläuchen (I,~35). Hier wie dort liegt das
illusionserregende Moment in einem Traum.
}

\def\pxxiiini{
Mit einem für die kindische Erzählungsweise typischen Eingang: {\itquoted\spanish Eraselo
que se era, en hora buena sea, el mal que se vaya, el buen que se venga,
á pesar de Menga. Erase un hongo y una honga etc.} Zur Quelle vgl.\ Clemencin,
{\it Don Quijote}~2, S.~20, Anm.~39.
}

\def\pxxiiinii{
Vgl.\ den Streit um den Saumsattel I,~45.
}

\def\pxxivni{
cf.~Cervantes (I,~52): {\itquoted\italian Forsi altro cantera con miglior plectro}
(statt {\itquoted\italian Forse
altro canterà con miglior plettro}, {\it Orl.~Fur.}~XXX,~16).
}

\def\pxxvni{
I,~13 und II,~32; cf.~Bojardo, {\it\italian Orlando innamorato} (lib.~I, canto~18):
{\obeylines\parindent=2\the\parindent%
\glqq Perchè ogni cavalier, ch'è senza amore
Se in vista è vivo, vivo è senza core.\grqq (Str.~46.)
}
}

\def\pxxvini{
cf.~Kraepelin, {\it Psychiatrie}~I, S.~134.
}

\def\pxxvinii{
Kraepelin, {\it Lehrbuch der Psychiatrie} S.~138.
}

\def\pxxviniii{
Wundt, {\it Phys.\ Psychologie}~III, S.~643: {\quoted Hallucinationen sind
Erinnerungsbilder, die sich von den normalen nur durch ihre Intensität unterscheiden.}
}

\def\pxxviniv{
I,~18.
}

\def\pxxviini{
Der Ausdruck {\quoted Hilfsillusion} ist insofern nicht ganz zutreffend, als es sich
hier nicht gemäss der oben gegebenen Definition um eine Sinnestäuschung handelt.
Ich habe ihn gewählt, weil ich die Hilfsillusion als einen Teil der dem Ganzen zu
Grunde liegenden Illusion ansehe.
}

\def\pxxviinii{
Ebenso I,~4. Es ist dies das einzige Beispiel einer akustischen Reizung
im {\it Don Quijote.}
}

\def\pxxviiini{
La Huerta, {\it Theatro Hespañol} 1735--88, berichtet einen ähnlichen Fall.
Bei einer Aufführung von {\it\spanish La niña de Gomez~Arias} von~Calderon hatten die
Alcaldes de Corte, welchen die Aufsicht über das Theater oblag, ihren Platz auf
der Bühne eingenommen und waren von einigen Alguacils begleitet. In der Scene
nun, wo Gomez~Arias das unglückliche Mädchen, das er verführt hat, den Mauren
verkaufen will, wurde einer der Alguacils so von der Lebendigkeit und Naturwahrheit
der Darstellung hingerissen, dass er mit gezogenem Schwert auf
den Schauspieler losging, der die Rolle des Gomez spielte, und ihn zur Flucht
zwang.
}

\def\pxxviiinii{
{\it Unterhaltungen mit Müller.}
}

\def\pxxixni{
{\spanish {\it La Celestina, Tragicomedia de Calisto y Melibea} por Fernando de~Rojas}
(her.\ von Don Marcelino Menendez y~Pelayo). Vigo~1899, 1900. I, S.~102.
}

\def\pxxixnii{
a.~a.~O.\ S.~110.
}

\def\pxxixniii{
Av.~Cap.~XXIV: {\itquoted\spanish cuyo valor conocieron tambien los griegos!}
}

\def\pxxixniv{
Canto XXXVII, Str.~5.
}

\def\pxxixnv{
Canto XX.
}

\def\pxxxni{
Schneegans, H., {\it Geschichte der grotesken Satire.} Strassburg~1895.
}

\def\pxxxnii{
Gaspary, {\it Geschichte der italienischen Literatur.} Strassburg~1888. S.~440.
}

\def\pxxxini{
cf.~{\it D.~Q.}\ I,~2: {\it{\quoted\spanish Apenas había el rubicundo Apolo tendido por la faz de
la ancha y espaciosa tierra las doradas hebras de sus hermosos cabellos} etc.}
I,~13: {\it{\quoted\spanish Mas apenas comenzó á descubrirse el día por los balcones
del oriente} etc.}
}

\def\pxxxiini{
cf.~Morel-Fatio, {\it\french Le Don Quichotte envisagé comme Peinture et Critique
de la société espagnole du XVI\sup{e} et du XVII\sup{e} siècle}
({\it\french Études sur l'Espagne}~I, 1895, S.~297--382).
}

\def\pxxxiiini{
{\quoted Das Gefühl der Komik entsteht dann infolge des Kontrastes von Zweck
und Realisierung, der zugleich ein solcher von Gross und Klein ist.} cf.~Lipps
a.~a.~O.\ S.~48.
}

\def\pxxxiiinii{
{\quoted Das Possenhafte ist eine niedere Art des Komischen oder besser eine
weniger feine Art, einen komischen Effekt hervorzubringen.} cf.~Lipps a.~a.~O.
S.~168 u.~169.
}

\def\pxxxiiiniii{
cf. S.~27~f.
}

\def\pxxxiiiniv{
Über die Komik der Selbsttäuschung vgl.~Klein, {\it Geschichte des Dramas,}
2.~Bd., S.~38~ff. und Wetz, {\it Die Anfänge der ernsten bürgerlichen Dichtung des
18.~Jahrhunderts.} Worms~1885. S.~72~ff.
}

\def\pxxxivni{
{\it Philosophie des Schönen,} S.~393~ff.
}

\def\pxxxvni{
In der Vorrede seiner Übersetzung.
}

\def\pxxxvnii{
{\quoted Sie (die komischen Figuren) stehen dem Helden in parodischer Weise
gegenüber und wiederholen die von idealen Motiven geleiteten Handlungen derselben
in einer niederen Sphäre, sodass {\emph erhaben} zur Karikatur wird.} --- Schack II, S.~253.
}

\def\pxxxvniii{
Cap.~XXXII.
}

\def\pxxxvini{
Wie bei dem oben citierten Certamen in Zaragoza.
}

\def\pxxxvinii{
Mariana, {\it\spanish contra los juegos publicos.} 1609.
}

\def\pxxxviini{
Dem Renaissancemenschen Bojardo musste diese ritterliche Tugend besonders
lächerlich erscheinen. Er satirisiert sie an mehreren Stellen, worunter
die Scene, wie Angelica Roland badet, als besonders grotesk hervorragt.
}

\def\pxxxviinii{
I,~30: \dots {\itquoted\spanish el asno callaba y se dejaba besar y acariciar de Sancho sin
responderle palabra alguna} und Av.\ Cap.~XXI: \dots {\itquoted\spanish subido en su rucio
sin permitirle decir en el camino palabra buena ni mala.}
}

\def\pxxxviiini{
Noch deutlicher ist die groteske Darstellung in Guevaras {\it\spanish Diablo cojuelo,} (1641)
(gegen Schneegans, a.~a.~O.\ S.~470, der den Spaniern die Kenntnis des grotesken
Stiles abspricht): z.~B. Traneo~II.: {\it{\quoted\spanish Pero ¿quien es aquella Habada con
camisa de muger que no solamente la cama le viene estrecha, sino la casa y Madrid, que haze
roncando mas ruido que la Bermuda y al parecer beue camaras de tinajas y come
gigotes de bobedas? \dots se piensa ir al Cielo derecha, que aunque pongan una garrucha
en la Estrella de Venus y un alçaprima en las siete Cabrillas, me parece
que será imposible que suba allá aquel tonel} etc.}
}

\def\pxxxviiinii{
Gayangos, {\it\spanish Libros de Caballerías} ({\it Amadis de Gaula} und {\it Esplandián\/}),
{\it Bibl.\ de aut.\ esp.}, vol.~40, 1857.
}

\def\pxxxixni{
Ähnliches kommt vielfach vor, vgl.\ {\it D.~Q.}\ I,~50 das Abenteuer von dem
kochenden See der sieben Feen und dazu die Anmerkung von Clemencin.
}

\def\pxxxixnii{
C.~Voretzsch, {\it Epische Studien} I, S.~132 führt noch andere Beispiele von
Torsperrungen an: die Schwertbrücke ({\it Lanzelot\/}), das Rad ({\it Wigalois\/}) und das
Fallgatter ({\it Ivain\/}).
}

\def\pxxxixniii{
Duran, {\it\spanish Romancero general} I, S.~243.
}

\def\pxxxixniv{
ib. S.~212.
}

\def\pxlini{
Von La~Fontaine bearbeitet in dem Conte {\it Le Muletier.}
}

\def\pxlinii{
M.~Landau, {\it Die Quellen des Dekameron.} Stuttgart~1884. 2.~Aufl. S.~76.
}

\def\pxliini{
Percy, {\it\english Reliques of ancient poetry.} London~1823. Vol.~III, S.~304--308. ---
John S.\ Roberts, {\it\english The legendary ballads of England and Scotland.} London.
S.~26--33.
}

\def\pxliinii{
Die Bilder sind nicht geschmackvoll, z.~B.: \dots {\it{\quoted\spanish vi en ese bello aspecto
\dots una esplendidisima mesa de regalados manjares para el gusto, pues le tuve y tengo
el mayor que jamas he tenido, en ver la virtud que resplandece en vuesa merced,
pan confortativo de mis desmayados alientos, acompañada de la sal de sus gracias,
y vino de su risueña afabilidad; si bien me acobarda el cuchillo del rigor} etc.}
}

\def\pxliiini{
In dem falschen {\it Guzman} (II,~6) sehen wir einen {\itquoted\spanish devoto de monjas} mit
einer Nonne und einer Dame darüber disputieren, ob in der Liebe die Hoffnung
oder der Besitz besser sei.
}

\def\pxlivni{
A.~Mussafia, {\it Studien zu den ma.~Marienlegenden.} Wien~1887--98. 5~Teile
(Sitzungsberichte der Akad.\ der Wissenschaften). --- Gröber, {\it Ein Marienmirakel}
(Festschrift für Wendelin Förster, 1902, S.~421~ff.). --- Heinrich Watenphul, {\it Die
Geschichte der Marienlegende von Beatrix der Küsterin.} Diss. Göttingen~1904.
}

\def\pxlivnii{
{\it\latin Dialogus miraculorum,} lib.~VII. No.~34. ed.~Strange, Köln~1851. Bd.~II,
S.~42,~43.
}

\def\pxlivniii{
Anonym, Vincentius Bellovacensis, Étienne de Besançon etc.
}

\def\pxlivniv{
Jean Mielot (15.~Jh.), Gautier de~Coincy, Jacques de~Vitry, ein Mirakelspiel etc.
}

\def\pxlivnv{
{\it\french Légende de sœur Béatrix (Contes de la veillée).} Paris~1838.
}

\def\pxlivnvi{
{\it\french Sœur Béatrice.} Paris~1901.
}

\def\pxlvni{
{\it\french Fabliaux et contes}~V.
}

\def\pxlvnii{
\dots {\itquoted\spanish si bien otra igual á ella en la sustancia tengo leida en el milagro
veinte y cinco de los noventa y nueve que de la Virgen sacratisima recogió en su
tomo de sermones el grande autor y maestro que por humildad quiso llamarse
el discípulo: libro bien conocido y aprobado.}
}

\def\pxlvniii{
{\it\latin Sermones Discipuli {\rm (Joannes Herolt sancti Dominici ordinis)} et de Tempore
et de Sanctis cum Exemplorum Promptuariis,}~1418. Herold ist geboren zu Kostnitz
oder Basel. Er nennt sich discipulus {\itquoted\latin quia in istis sermonibus non subtilia per
modum magistri vel doctoris, sed simplicia per modum discipuli conscripsi
et collegi}.
}

\def\pxlvniv{
cf.~Oeuvres de Rutebeuf ed.~par Achille Jubinal,~1839. I,~S.~302: {\it\french Du
Secrestain et de la Famme au Chevalier.}
}

\def\pxlvini{
{\swedish In {\it Det norske oldscrift selkabs samlinger~XII. Norröne skrifter af
religiöst inhold: Mariu Saga. Legender om jomfru Maria og hendes jertegn.}
efter gamle handskrifter udgivne af C.~R.~{\emph Unger,} andet hefte, Christiania~1869.
S.~514--21.} --- Watenphul, a.~a.~O., S.~56.
}

\def\pxlvinii{
W.~J.~A.~Jonkbloet, {\it Beatrijs.} 1.~Aufl. Amsterdam~1846. 2.~Aufl.~zusammen
mit Carel ende Elegast. Amsterdam~1859.
}

\def\pxlviniii{
cf.~Watenphul, a.~a.~O., S.~29.
}

\def\pxlviniv{
Dieselbe Legendenform findet sich als Ballade in Montanus (Vincenz von
Zuccalmaglio), {\it Die Vorzeit der Länder Kleve-Mark, Jülich-Berg und Westphalen,
Solingen u.~Gummershausen}~1837. I,~S.~350~ff. {\it Bergische Klostersage aus dem
13.~Jahrhundert.} cf.~Watenphul, a.~a.~O., S.~73.
}

\def\pxlviini{
a.~a.~O.\ S.~81.
}

\def\pxlviinii{
Fitzmaurice-Kelly, {\it\english Life of Cervantes.} London~1892.
}

\def\pliini{
cf.~{\it Orl.~Fur.}\ Canto~XXXVI, Str.~78~ff.: Das Zelt Melissas.
}

\def\pliiini{
{\it\french Roland l'Amoureux} 1717--21.
}

\def\pliiinii{
{\it Guzman d'Alfarache} 1732. --- {\it Estebanille Gonzalès.} Paris, Ganneau~1732.
}

\def\plvini{
Sorel, {\it\french Le Berger extravagant,} verspottet sie, indem er aus den
metaphorischen Bestandteilen ein Bild zusammensetzen lässt. Vgl.~auch {\it D.~Q.}\ I,~13:
{\itquoted\spanish sus cabellos son oro, su frente campos eliseos, sus cejas arcos
del cielo, sus ojos
soles, sus mejillas rosas, sus labios corales, perlas sus dientes, alabastro su cuello,
mármol su pecto, marfil sus manos, su blancura nieve, y las partes que á la vista
humana encubrió la honestidad son tales segun yo pienso y entiendo que sólo la
discreta consideracion puede encarecerlas y no compararlas.} Zu den letzten Worten
cf.~{\it Tirante el Blanco,} I,~17.
}

\def\plvinii{
1616; frz.~Übers.~von J.~Pradelle Baudouin, 1646 u.~1658.
}

\def\plviini{
Bentivoglio, {\it Storia de las guerras di Fiandra.} 1633--1639. Bruslé de
Montpleinchamp, {\it\french Histoire de l'archiduc Albert, prince souverain de la Belgique.}
Cologne~1693, war mir nicht zugänglich.
}

\def\plviinii{
In den Schicksalen der Familie de Leyva.
}

\def\plviiniii{
Huet d'Avranches in seinem {\it\french Traité sur l'origine des romans,} mit dem er
die {\it Zayde} von Mme~de La~Fayette begleitete (1670).
}

\def\plviiini{
Es ist schade, dass der 3.~Band von Rius (1905): {\quoted Urteile über Cervantes}
allzu sehr die Panegyriker, zu wenig die Kritiker berücksichtigt. Wir finden bei
ihm weder Sorel noch Lesage.
}

\def\plviiinii{
a.~a.~O., S.~788.
}

\def\plixni{
Clerville in {\it\french Le Gascon extravagant,} 1639, einer Nachahmung des {\it\french Berger
extravagant} und des {\it Don Quijote} (cf.~Körting, a.~a.~O., II, S.~98), l.~I, S.~66:
{\itquoted\french Elle
lisait le Don Quichot de la Manche qu'elle avoit pris sur ma table, et s'y plaisait
tellement qu'encore que nous eussions demeuré prés de trois quarts-d'heure, sa
Suivante et moy, elle ne s'imaginoit pas qu'il y eust un moment que nous fussions
ensemble.}
}

\def\plixnii{
a.~a.~O.: {\itquoted\french Miguel de Cervantes, l'un des plus beaux esprits que l'Espagne ait
produit} \dots
}

\def\plixniii{
{\it\french Continuation des Mémoires de Sallengre.} t.~VI.
}

\def\plxni{
Vgl.\ Goethes Äusserung über Shakespeare: {\quoted Der Dichter lässt seine
Personen jedesmal das reden, was eben an dieser Stelle gehörig, wirksam und
gut ist, ohne sich viel und ängstlich zu bekümmern und zu kalkulieren, ob diese
Worte vielleicht mit einer anderen Stelle in scheinbaren Widerspruch geraten
können.} ({\it Goethe-Eckermann,} 18.~April 1827.)
}

\def\plxnii{
Jedenfalls waren die Gefangenen hintereinander an einer Kette ({\it cadena\/})
befestigt, die durch an den Handschellen ({\it esposas\/}) befindliche Ringe lief, so dass
die Kette nur bei dem vordersten gelöst zu werden brauchte, damit die ganze
Reihe frei wurde.
}

\def\plxniii{
Charles and Mary Cowden Clarke, {\it The Shakespeare Key.} London~1879.
S.~105: {\it Dramatic Time.}
}

\def\plxini{
Über den Mangel an Erfindung bei Lesage cf.~Claretie, {\it\french Essai sur Lesage
Romancier.} Paris~1890. S.~187~ff.
}

\def\plxinii{
{\it L.~S.}\ Cap.~47.
}

\def\plxiini{
Auch hier hat wieder das Beispiel des Cervantes gewirkt. cf.~II, 45, 47, 49,~51.
}

\def\plxiinii{
{\it LS,} Cap.~29.
}

\def\plxiiniii{
Chap.~XIV.
}

\def\plxiiniv{
Baret, Eug., {\it\french De l'Amadis de Gaule et de son influence sur les mœurs et
la littérature au XVI\sup{e} et au XVII\sup{e} siècle.} Paris~1853.
}

\def\plxiinv{
{\it Gil Blas,} Livre~X, Ch.~VII.
}

\def\plxiiini{
{\it Gil Blas,} Livre~X, Ch.~VIII.
}

\def\plxiiinii{
{\it Le Bachelier de Salamanque,} Livre~I, Ch.~VII.
}

\def\plxiiiniii{
{\it Le Diable Boiteux,} édition augmentée, 1726, Ch.~VII.
}

\def\plxiiiniv{
{\it Gil Blas,} Livre~VII, Ch.~II.
}

\def\plxiiinv{
{\it Gil Blas,} Livre~VIII, Ch.~IX.
}

\def\plxiiinvi{
Im {\it Gil Blas} (II, IX) beklagt der Onkel Diegos, dass man die Pastorale
nicht mehr wie früher schätze.
}

\def\plxiiinvii{
Honoré d'Urfé, {\it Astrée,} 1610, 1619, 1627.
}

\def\plxivni{
{\it\french La Valise trouvée} (1740) S.~3: {\itquoted\french Parbleu! s'écria
le Marquis, à peu près
l'avanture de Don Quichotte et de Sancho dans la Montagne noire. Voyons un peu
si cette Valise renferme autant d'écus que celle de Cardenio.}
}

\def\plxivnii{
Chap.~XVII.
}

\def\plxvni{
Dem {\quoted weiblichenDon Quichotte} begegnet man auch sonst in der Litteratur.
In {\french Adrien-Thomas Perdon de Subligny, {\it La Fausse Clélie}} (1670, 1671, 1680,
1712, 1718) wird die Heldin über der Lektüre der 8000 Seiten füllenden {\it\french Histoire Romaine}
verrückt. cf.~Körting, a.~a.~O.\ II, S.~273. Der Roman {\it The Female Quixote} von
Mrs.~Lennox~(1752) ist eine Satire auf die Romane der Scudérys. Die Heldin
Arabella, eine Dame von Geburt, ausgestattet mit liebenswürdigen Eigenschaften,
welche jedoch, da ihr Vater sie in völliger Abgeschiedenheit von der Welt
aufgezogen und sie fortwährend die Romane der Scudérys gelesen hat, zuletzt die
Ereignisse dieser Dichtungen für wahr hält und ihr Benehmen danach einrichtet.
Sie bildet sich ein, dass jeder Mann in sie verliebt sei, und schwebt in beständiger
Furcht, entführt zu werden. Den Gärtner ihres Vaters hält sie für einen
verkleideten Prinzen, und entlässt einen vernünftigen Liebhaber, weil er in dem Kodex
der heroischen Galanterie nicht gehörig Bescheid weiss. --- cf.~Dunlop-Liebrecht,
S.~334. --- Die dem Sancho Panza entsprechende Vertraute Lucia hat nichts von
den komischen Seiten ihres Pendants.

Dieser Roman wurde 1773 in das Französische übersetzt, 1754~in das Deutsche
und 1808 in das Spanische. --- Rius, a.~a.~O.\ II, S.~304.
}

\def\plxvnii{
Livre~I, Ch.~IX.
}

\def\plxvniii{
cf.~Körting, a.~a.~O.\ II, S.~86~ff.
}

\def\plxvini{
Körting, a.~a.~O.\ II, S.~99.
}

\def\plxvinii{
Der ursprüngliche Titel lautet {\it Un Don Quichotte moderne.} Erst die durchgesehene
Ausgabe von~1737 trägt den Titel {\it\french Pharsamond ou les folies romanesques.}
}

\def\plxviini{
Schneegans, a.~a.~O., S.~91.
}

\def\plxviinii{
Schneegans, a.~a.~O., S.~174~ff.
}
